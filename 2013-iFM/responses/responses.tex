\documentclass{response}

\papertitle{Progress Concerns as Design Guidelines}
\paperauthor{Simon Hudon and Thai Son Hoang}

\begin{document}

 
\comment{1}{1. The availability of the proof-rules
 
In Section 5 (Conclusion), the authors say that "We expect to have
more refinement rules to complement the current set of rules." So, it
would be better to explain how powerful the proof-rules currently
given in Theorems 1-6 are. For example, the assumptions in Theorems 6
and 7 are sufficient conditions, but are not necessary. Therefore, if
an example which cannot be proved by the current rules is shown, it
would be good information for readers. If current rules are already
powerful enough to prove most of practical cases, it should be briefly
stated.
 
Note: this comment does not request to construct the sound and
complete rules. The reviewer also think that it is important to give
some useful and practical rules, as given in this paper.
}
 
\response{We mentioned that the rules are sufficient for us to develop
several examples of different size.}

\comment{1}{2. Typos etc.}
 
\response{The typos have been fixed. Thank you very much.}

\comment{1}{
\begin{verbatim}
(5) P.5, L.-11:
     "... => X; s.t.e.v>"
==> "... => X; s.t.e.v'>"
\end{verbatim}
}
\response{$v$ is correct. The fresh variable $e$ stores the before
  value $v$ and $v$ in $s.t.e.v$ represents the current (after) value
  of $v$.}

\comment{2}{
One point: I was mystified by the title. Indeed the paper does not really seem to offer “Design Guidelines” at all. 
}

\response{Hmmm, we do not mean design guidelines as some generic
  guidelines.}

\comment{2}{
  Minor typos
}

\response{The typos have been fixed. Thank you very much.}


\comment{2}{
  Pg 13, paragraph beginning “We want to…” This was a slightly confusing discussion, because of the amount of “infinitely oftens” going on. Can you try to clarify? 
}

\response{
  Can we rephrase this to have a clearer explanation.
}

\comment{2}{
  The paper is over length. Please don’t cut the conclusions (which are useful). I would try condensing the initial presentation of the example (put the assumptions in a table, maybe? Condense the proof of Theorem 2 as well?
}

\response{
  We reduce the size of the paper.
}


\comment{3}{
There is a lot in the paper. There is significant depth in the technical detail and sometimes it would have helped the reader to have some illustrating examples as we went through the paper. It is very important on page 5 to understand (14) before moving on. It was not so clear whether c in the c.t.v was a Cpred.
An example is needed by the time we get to section 3.1.
}

\responses{
  We should say that g, c, f are state predicates.
  Slipping in some examples will cause some problem with the size of
  the paper.
}

\comment{3}{
  Minor typos
}

\responses{The typos have been fixed. Thank you very much.}

\comment{3}{
\begin{verbatim}
Page 3 Defn 6 For all predicates
Page 3 Defn 8 For all state predicates
\end{verbatim}
}

\response{
  Changed to \emph{For any predicate} or \emph{For any state
    predicate} (globally).
}

\comment{3}{
\begin{verbatim}
ASM 1 There is one
ASM 4 There is a light
\end{verbatim}
}

\response{
  \emph{There is/are} here are both acceptable. I changed to the
  reviewers' comments.
}
\end{document}



