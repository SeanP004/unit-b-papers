\documentclass{response}

\papertitle{Progress Concerns as Design Guidelines}
\paperauthor{Simon Hudon and Thai Son Hoang}

\begin{document}

 
\comment{1}{1. The availability of the proof-rules
 
In Section 5 (Conclusion), the authors say that "We expect to have
more refinement rules to complement the current set of rules." So, it
would be better to explain how powerful the proof-rules currently
given in Theorems 1-6 are. For example, the assumptions in Theorems 6
and 7 are sufficient conditions, but are not necessary. Therefore, if
an example which cannot be proved by the current rules is shown, it
would be good information for readers. If current rules are already
powerful enough to prove most of practical cases, it should be briefly
stated.
 
Note: this comment does not request to construct the sound and
complete rules. The reviewer also think that it is important to give
some useful and practical rules, as given in this paper.
}
 
\response{We rephrase the paragraph to state more precisely that we
  expect to have separate rules for data refinement.}

\comment{1}{2. Typos etc.}
 
\response{The typos have been fixed. Thank you very much.}

\comment{1}{
\begin{verbatim}
(5) P.5, L.-11:
     "... => X; s.t.e.v>"
==> "... => X; s.t.e.v'>"
\end{verbatim}
}
\response{$v$ is correct. The fresh variable $e$ stores the before
  value $v$ and $v$ in $s.t.e.v$ represents the after value
  of $v$.}

\comment{2}{
One point: I was mystified by the title. Indeed the paper does not really seem to offer “Design Guidelines” at all. 
}

\response{We changed the title to ``Systems Design Guided by Progress Concerns''}

\comment{2}{
  Minor typos
}

\response{The typos have been fixed. Thank you very much.}


\comment{2}{
  Pg 13, paragraph beginning “We want to ...” This was a slightly confusing discussion, because of the amount of “infinitely oftens” going on. Can you try to clarify? 
}

\response{
  We rephrase the paragraph to explain our point better.
}

\comment{2}{
  The paper is over length. Please don’t cut the conclusions (which are useful). I would try condensing the initial presentation of the example (put the assumptions in a table, maybe? Condense the proof of Theorem 2 as well?
}

\response{
  We have reduced the size of the paper to 15 pages.
}


\comment{3}{
There is a lot in the paper. There is significant depth in the
technical detail and sometimes it would have helped the reader to have
some illustrating examples as we went through the paper. It is very
important on page 5 to understand (14) before moving on.
}

\response{
  A running example with the introduction of Unit-B certainly improves
  the presentation. However, given the current size of the paper current, we
  would like to keep the space for more important details about our
  method and the main case study.
}

\comment{3}{
 It was not so clear whether c in the c.t.v was a Cpred.
}

\response{
  We mention that g,c,f are state predicates.
}

\comment{3}{
An example is needed by the time we get to section 3.1.
}
\response{Space problem again.}

\comment{3}{
  Minor typos
}

\response{The typos have been fixed. Thank you very much.}

\comment{3}{
\begin{verbatim}
Page 3 Defn 6 For all predicates
Page 3 Defn 8 For all state predicates
\end{verbatim}
}

\response{
  Changed to \emph{For any predicate} or \emph{For any state
    predicate} (globally).
}

\end{document}



