\documentclass{response}

\papertitle{Progress Concerns as Design Guidelines}
\paperauthor{Simon Hudon and Thai Son Hoang}

\begin{document}

 
\comment{1}{1. The availability of the proof-rules
 
In Section 5 (Conclusion), the authors say that "We expect to have
more refinement rules to complement the current set of rules." So, it
would be better to explain how powerful the proof-rules currently
given in Theorems 1-6 are. For example, the assumptions in Theorems 6
and 7 are sufficient conditions, but are not necessary. Therefore, if
an example which cannot be proved by the current rules is shown, it
would be good information for readers. If current rules are already
powerful enough to prove most of practical cases, it should be briefly
stated.
 
Note: this comment does not request to construct the sound and
complete rules. The reviewer also think that it is important to give
some useful and practical rules, as given in this paper.
}
 
\response{We mentioned that the rules are sufficient for us to develop
several examples of different size.}

\comment{1}{2. Typos etc.}
 
\response{The typos have been fixed. Thank you very much.}

\comment{1}{
\begin{verbatim}
(5) P.5, L.-11:
     "... => X; s.t.e.v>"
==> "... => X; s.t.e.v'>"
\end{verbatim}
}
\response{$v$ is correct. The fresh variable $e$ stores the before
  value $v$ and $v$ in $s.t.e.v$ represents the current (after) value
  of $v$.}

\begin{verbatim}


\end{verbatim}
\end{document}

