%%%% This is a fix for amsmath's warning.
% Package amsmath Warning: Unable to redefine math accent \vec.
\newcommand\PREAMBLE{}
\let\accentvec\vec
\documentclass{llncs}
\let\spvec\vec
\let\vec\accentvec

\author{Simon Hudon~\inst{1} \and Thai Son Hoang~\inst{2}}
\tocauthor{Hudon and Hoang}
\authorrunning{Hudon and Hoang}
\institute{
  Department of Computer Science,  York University, Toronto, Canada \\
  \email{simon@cse.yorku.ca}
  \and
  Institute of Information Security, ETH-Zurich, Switzerland \\
  \email{htson@inf.ethz.ch}
}

%\title{Progress-Preserving Refinement}
\title{Development of a Signal Control System Guided by Progress Concerns}
\date{\today}

% Theorem package
\newcounter{def}
\newtheorem{Definition}[def]{Definition}
\newcommand{\Defn}{Definition}
\newcounter{ptl}
\newtheorem{Postulate}[ptl]{Postulate}
\newcommand{\Post}{Postulate}
\newcounter{exp}
\newtheorem{Example}[exp]{Example}
\newcommand{\Exmp}{Example}
\newcounter{thm}
\newtheorem{Theorem}[thm]{Theorem}
\newcommand{\Thm}{Theorem}

\usepackage{abbrev}

% Some standard LaTeX packages.
\usepackage{xspace}
\usepackage{color}
\usepackage{datetime}
\usepackage{space-llncs}

% Package for common math symbols.
\usepackage{amsmath}
\usepackage{amssymb}

% Package for customized spacings
\usepackage{space}

% Package for reference
\usepackage{varioref}

% Use Times as the main fonts
\usepackage{times}

\usepackage{wrapfig}
\usepackage{reasoning}

% Use strike through text for the todos
\usepackage[normalem]{ulem}

% TODOs
\usepackage{todo}

% Package for calculational proofs
\usepackage{calculation}

% Package for computation calculus
\usepackage{comp-cal}

% Package for relational calculus
\usepackage{rel-cal}

% Package for xspace
\usepackage{xspace}

% Package for UNITY
\usepackage{unity}

% Package for B symbols
\usepackage{bsymb}

% Package for Event-B
\usepackage[color]{eventB}

% Package for TLA
\usepackage{tla}

% Package for Unit-B
\usepackage{unitb}

% Package for tikz
\usepackage{tikz}
\usetikzlibrary{snakes,arrows}

% Package for requirement document
\usepackage[compact]{reqdoc}
%%%% Some macros used in formal models.

\usepackage[plainpages=false]{hyperref}
\hypersetup{
  bookmarks=true,
  pdftitle={},
  pdfauthor={Simon Hudon and Thai Son Hoang,
    Department of Computer Science, York University, Toronto, Canada
    and
    Institute of Information Security, ETH Zurich, Switzerland
  },
  pdfsubject={Refinement, Progress},
  pdfkeywords={Refinement, Progress, Temporal Logic},
  colorlinks=true
}

\begin{document}

\section{Initial Model}
\label{sec:initial-model}

\newBset[TRAIN]{TRN}
\newBvrb[trains]{trns}
\newBpar[train]{t}
\newBevt{arrive}
\newBevt{depart}
\newBinv[invTrainType]{inv0\_1}

\begin{Bcode}
  $
  \carriersets{
    \TRAIN
  }
  $
  \Bhspace
  $ \variables{\trains} $
  \Bhspace
  $
  \invariants{
    \invTrainType: & \trains \subseteq \TRAIN
  }
  $
  \Bvspace
  $
  \initialisation{
    \trains = \emptyset
  }
  $
  \Bvspace
  $
  \ubevent{\arrive}
  {\train}
  {\train \in \TRAIN}
  {}
  {}
  {\trains \bcmeq \trains \bunion \{\train\}}
  $
  \Bhspace
  $
  \ubevent{\depart}{\train}{\train \in \TRAIN}{\train \in \trains}{}{\trains \bcmeq \trains \setminus \{\train\}}
  $
  \Bvspace
  $
  \properties{
    \Binv{prg0\_2}: & \tr \train \in \trains \\
    \Bthm{prg0\_1}: & \train \in \trains \leadsto t \notin \trains
  }
  $
\end{Bcode}

\subsection{First Refinement}
\label{sec:first-refinement}

\newBset[BLOCK]{BLK}
\newBcst{Entry}
\newBcst[PLATFORM]{PLF}
\newBcst{Exit}
\newBaxm[axmBlockType]{axm1\_1}
\newBaxm[axmPlatformNonEmpty]{axm1\_2}
\newBvrb[location]{loc}
\newBinv[invLocationType]{inv1\_1}
\newBevt{moveout}
\newBevt{movein}

\begin{Bcode}
  $
  \carriersets{\BLOCK}
  $
  \Bhspace
  $
  \constants{\Entry, \PLATFORM, \Exit}
  $
  \Bvspace
  $
  \axioms{
    \axmBlockType: & \partition(\BLOCK, \{\Entry\}, \PLATFORM,
    \{\Exit\}) \\
    \axmPlatformNonEmpty & \PLATFORM \neq \emptyset
  }
  $
  \Bvspace
  $
  \variables{\trains, \location}
  $
  \Bhspace
  $
  \invariants{
    \Binv{inv1\_1}: & \location \in \trains \tfun \BLOCK
  }
  $
  \Bvspace
  $
  \initialisation{
    \trains = \emptyset \wide\land \\
    \location = \emptyset
  }
  $
  \Bvspace
  $
  \ubevent{\arrive}{\train}{\train \in \TRAIN}{}{}{
    \trains \bcmeq \trains \bunion \{\train\} \\
    \location(\train) \bcmeq \Entry
  }
  $
  \Bhspace
  $ \ubevent{\depart}{\train}{ \train \in \trains \land
    \location.\train = \Exit } { \train \in \trains \land
    \location.\train = \Exit} {} {\trains \bcmeq \trains \setminus
    \{\train\} \\ \location \bcmeq \{\train\} \domsub \location} $
  \Bvspace
    $
  \ubevent{\moveout}{\train}
  {\train \in \trains \land \location.\train \in \PLATFORM}
  {\train \in \trains \land \location.\train \in \PLATFORM}
  {}
  {\location.\train \bcmeq \Exit}
  $
  \Bvspace
  $
  \ubevent{\movein}{\train}
  {\train \in \trains \land \location.\train = \Entry}
  {\train \in \trains \land \location.\train = \Entry}
  {}
  {\location \bcmsuch \qexists{p}{p \in \PLATFORM}{\location'
    = \location \ovl \{\train \mapsto p\}}}
  $
  \Bvspace
  $
  \properties{
    \Binv{prg1\_9}: & \tr  \train \in \trains \land \location.\train
    \neq \Exit \land \location.\train \notin \PLATFORM \\
    \Binv{un1\_8}: & \train \in \trains \land \location.\train
    \neq \Exit \land \location.\train \notin \PLATFORM ~\un~ \train \in
    \trains \land \location.\train \in \PLATFORM \\
    \Bthm{prg1\_7}: & \train \in \trains \land \location.\train
    \neq \Exit \land \location.\train \notin \PLATFORM ~\leadsto~ \train \in
    \trains \land \location.\train \in \PLATFORM \\
    \Binv{prg1\_6}: & \tr \train \in \trains \land \location.\train \in
    \PLATFORM \\
    \Binv{un1\_5}: & \train \in \trains \land \location.\train \in \PLATFORM ~\un~
    \train \in \trains \land \location.\train = \Exit \\
    \Bthm{prg1\_4}: & \train \in \trains \land \location.\train
    \in \PLATFORM ~\leadsto~ \train \in
    \trains \land \location.\train = \Exit \\
    \Bthm{prg1\_3}: & \train \in \trains \land \location.\train
    \neq \Exit ~\leadsto~ \train \in
    \trains \land \location.\train \in \PLATFORM \\
    \Bthm{prg1\_2}: & \train \in \trains \land \location.\train \neq \Exit ~\leadsto~ \train \in \trains \land
    \location.\train = \Exit \\
    \Binv{un1\_2}: & \train \in \trains \land \location.\train =
    \Exit ~\un~ \spneg (\train \in \trains)\\
    \Bthm{prg1\_1}: & \train \in \trains ~\leadsto~ \train \in \trains
    \land \location.\train = \Exit
  }
  $
\end{Bcode}

Reasoning remarks:
\begin{center}
  \begin{tabular}{|l|l|}
    \hline
    Consequence & Antecedents \\
    \hline
    Coarse-schedule strengthening \depart & \Bthm{prg1\_1},
    \Binv{un1\_2} \\
    \Bthm{prg1\_1} & \Bthm{prg1\_2} (Split-off-skip) \\
    \Bthm{prg1\_2} & \Bthm{prg1\_3}, \Bthm{prg1\_4} (transitivity) \\
    \Bthm{prg1\_4} & \Binv{un1\_5}, \Binv{prg1\_6} (ensure-rule) \\
    \Binv{prg1\_6} & implemented by $\moveout.\train$ (transient-rule) \\
    \Bthm{prg1\_3} & \Binv{prg1\_7}  (Split-off-skip) \\
    \Bthm{prg1\_7} & \Binv{un1\_8}, \Binv{prg1\_9} (ensure-rule) \\
    \Binv{prg1\_9} & implemented by $\movein.\train$ (transient-rule) \\
    \hline
  \end{tabular}
\end{center}

\subsection{Second Refinement}
\label{sec:second-refinement}

\begin{Bcode}
  $
  \variables{\trains, \location}
  $
  \Bvspace
  $
  \invariants{
    \Binv{inv2\_1}: & \qforall{\train_1,\train_2}{\train_1, \train_2 \in \trains \land \location.\train_1 =
  \location.\train_2}{\train_1 = \train_2}
  }
  $
  \Bvspace
  $
  \ubevent{\arrive}{\train}{\train \notin \trains \\ \Entry \notin \ran(\location) }{}{}{
    \trains \bcmeq \trains \bunion \{\train\} \\
    \location(\train) \bcmeq \Entry
  }
  $
  \Bhspace
  $ \ubevent{\depart}{\train}{ \train \in \trains \land
    \location.\train = \Exit } { \train \in \trains \land
    \location.\train = \Exit} {} {\trains \bcmeq \trains \setminus
    \{\train\} \\ \location \bcmeq \{\train\} \domsub \location} $
  \Bvspace
    $ \ubevent{\moveout}{\train}{ \train \in \trains \land
      \location.\train \in \PLATFORM \land \\
      \Exit \notin \ran.\location
    }{\train \in \trains \land \location.\train \in \PLATFORM} {\Exit
      \notin \ran.\location} {\location.\train \bcmeq \Exit} $
    \Bvspace
  $
  \ubevent{\movein}{\train}
  {\train \in \trains \land \location.\train = \Entry \land
    \qexists{p}{p \in \PLATFORM}{p \notin \ran.\location}
  }
  {\train \in \trains \land \location.\train = \Entry  \land     \qexists{p}{p \in \PLATFORM}{p \notin \ran.\location}}
  {}
  {\location \bcmsuch \qexists{p}{p \in \PLATFORM \setminus 
    \ran.\location}{\location'
    = \location \ovl \{\train \mapsto p\}}}
  $
  \Bvspace
  $
  \properties{
    \Binv{prg2\_9}: & \tr \train \in \trains \land \location.\train =
    \Entry \land \qforall{p}{p \in \PLATFORM}{p \in \ran.\location} \\
    \Binv{un2\_8}: & \train \in \trains \land \location.\train =
    \Entry \land \qforall{p}{p \in \PLATFORM}{p \in \ran.\location} \\
    & \WIDE\un \train \in \trains \land \location.\train = \Entry
    \land     \qexists{p}{p \in \PLATFORM}{p \notin \ran.\location} \\
    \Bthm{prg2\_7}: & \train \in \trains \land \location.\train =
    \Entry \land \qforall{p}{p \in \PLATFORM}{p \in \ran.\location} \\
    & \WIDE\leadsto \train \in \trains \land \location.\train = \Entry
    \land     \qexists{p}{p \in \PLATFORM}{p \notin \ran.\location} \\
    \Bthm{prg2\_6}: & \train \in \trains \land \location.\train =
    \Entry \\
    & \WIDE\leadsto \train \in \trains \land \location.\train = \Entry
    \land     \qexists{p}{p \in \PLATFORM}{p \notin \ran.\location} \\
    \Binv{un2\_5}: & \train \in \trains \land \location.\train = \Entry
    \land     \qexists{p}{p \in \PLATFORM}{p \notin \ran.\location} \\
    & \Wide{\un} \spneg (\train \in \trains \land \location.\train =
    \Entry) \\
    \Binv{prg2\_4}: & \tr \Exit \in \ran.\location \\
    \Bthm{prg2\_3}: & \one \leadsto \Exit \notin \ran.\location \\
  }
  $
\end{Bcode}
% Event \arrive is refined straightforward
% \end{Bcode}

Reasoning remarks
\begin{center}
  \begin{tabular}{|l|l|}
    \hline
    Consequence & Antecedents \\
    \hline
    Fine-schedule strengthening \moveout & \Bthm{prg2\_3} \\
    \Bthm{prg2\_3} & \Binv{prg2\_4} ($\tr$ property) \\
    \Binv{prg2\_4} & implemented by $\depart.(\location^{-1}.\Exit)$
    (transient-rule) \\
    Coarse-schedule replacing \movein & \Binv{un2\_5}, \Bthm{prg2\_6}
    \\
    \Bthm{prg2\_6} & \Bthm{prg2\_7} (Split-off-skip) \\
    \Bthm{prg2\_7} & \Binv{un2\_8}, \Binv{prg2\_9} (ensure-rule) \\
    \Binv{prg2\_9} & implemented by \moveout (transient-rule)
    \hline
  \end{tabular}
\end{center}


Finally we strengthen the guard of \movein and subsequently strengthen
its coarse-schedule (see Figure~\ref{fig:2nd-ref}).  We apply
Theorem~\ref{thm:ref-rep-crs} (coarse-schedule replacing) \movein.
The detailed proofs is omitted here.

\subsection{Third Refinement}
\label{sec:third-refinement}

\newBset{COLOR}
\newBcst[GREEN]{GR}
\newBcst[RED]{RD}
\newBvrb[signal]{sgn}
\newBevt[ctrlplf]{ctrl\_platform}
\newBpar[platform]{p}

In this refinement, we introduce the signals associated with different
blocks within the network (\ref{asm:entry-signal},
\ref{asm:platform-signal}, \ref{asm:obey-signal}, and
\ref{asm:signal-to-red}).  Variable \signal is introduced to denote
the value of the signal associated with different blocks together with
the following invariants.  We focus on the controlling of the platform
signals here.  In particular, invariants \Binv{inv3\_2} and
\Binv{inv3\_3} state that if a platform signal is green (\GREEN) then
the exit block is free and the other platform signals are red (\RED).
  \begin{Bcode}[\footnotesize]
    $ \invariants[false]{ \Binv{inv3\_1}: & \signal \in \{\Entry\}
      \bunion \PLATFORM \tfun
      \COLOR \\
      \Binv{inv3\_2}: & \qforall{p}{p \in \PLATFORM \land \signal.p =
        \GREEN}{\Exit \notin \ran.\location} \\
      \Binv{inv3\_3}: & \qforall{p,q}{p, q \in \PLATFORM \land
        \signal.p = \signal.q = \GREEN}{p = q} } $
  \end{Bcode}

We refine the \moveout event using the platform signal as shown in Figure~\ref{fig:3rd-ref}.
\begin{figure}[!htbp]
  \centering
  \begin{Bcode}[\scriptsize]
    $
    \ubevent{\moveout}{\train}{
      \train \in \trains \land
      \location.\train \in \PLATFORM \land \\
      \signal.(\location.\train) = \GREEN
    }{\train \in \trains \land \location.train \in \PLATFORM \land \\
    \signal.(\location.\train) = \GREEN}{}{\location.\train \bcmeq
    \Exit \\
    \signal.(\location.\train) \bcmeq \RED
  }
  $
  \Bhspace
  $
  \ubevent{\ctrlplf}{\platform}{
    \platform \in \PLATFORM \land
    \platform \in \ran.\location \land
    \Exit \notin \ran.\location \land \\
    \qforall{q}{q \in \PLATFORM}{\signal.q = \RED}
  }
  {
    \platform \in \PLATFORM \land
    \platform \in \ran.\location \land
    \signal.\platform = \RED
  }
  {
    \Exit \notin \ran(\location) \land
    \qforall{q}{q \in \PLATFORM \land q \neq \platform}{\signal.q = \RED}
  }
  {\signal.\platform \bcmeq \GREEN}
  $
\end{Bcode}
\vspace{-4ex}
\caption{Events of the third refinement}
\label{fig:3rd-ref}
\end{figure}
The refinement of \moveout is justified by applying
Theorem~\ref{thm:ref-rep-crs} (coarse-schedule replacing) and
Theorem~\ref{thm:remove-fns} (fine-schedule removing).  In particular,
replacing the coarse-schedule requires the following progress
property.
\begin{equation}
  \label{eq:99}
  \train \in \trains \land \location.\train \in \PLATFORM
  \wide\leadsto \train \in \trains \land \location.\train \in
  \PLATFORM \land \signal.(\location.\train) = \GREEN
  \tag{\Binv{prg3\_4}}
\end{equation}
which can be implemented by
\begin{equation}
  \label{eq:35}
  \tr \train \in \trains \land \location.\train \in
  \PLATFORM \land \signal.(\location.\train) = \RED
  \tag{\Binv{prg3\_5}}
\end{equation}

In order to satisfy \ref{eq:35}, we introduce a new event
\ctrlplf for the controller to change a platform signal to green according to
Theorem~\ref{thm:transient} (see Figure~\ref{fig:3rd-ref}).
% \begin{Bcode}[\footnotesize]
%   $
%   \ubevent{\ctrlplf}{\platform}{
%     \platform \in \PLATFORM \land
%     \platform \in \ran.\location \land
%     \Exit \notin \ran.\location \land
%     \qforall{q}{q \in \PLATFORM}{\signal.q = \RED}
%   }
%   {
%     \platform \in \PLATFORM \land
%     \platform \in \ran.\location \land
%     \signal.\platform = \RED
%   }
%   {
%     \Exit \notin \ran(\location) \land
%     \qforall{q}{q \in \PLATFORM \land q \neq \platform}{\signal.q = \RED}
%   }
%   {\signal.\platform \bcmeq \GREEN}
%   $
% \end{Bcode}
This event \ctrlplf is a specification for the system to control the
platform signals preserving both safety and liveness properties of the
systems.  In particular, the scheduling information states that if (1)
a platform is occupied and the platform signal is \RED infinitely long
and (2) the exit block is unoccupied and the other platform signals
are all \RED infinitely often, then the system should eventually set
this platform signal to \GREEN.

The refinement of event \movein and how the entry signal is controlled
is similar and omitted.

\end{document}
%%% Local Variables: 
%%% mode: latex
%%% TeX-master: t
%%% End: 
