\section{Example: Developing a Mutual Exclusion Algorithm}
\label{sec:example}
\todo{3 pages}

We chose the problem of mutual exclusion to illustrate the use of \unitb for two reasons.  First, it features contention between processes over some share resource, which illustrates the strengths of strong fairness.  Second, the problem of mutual exclusion is the simplest of such problems which allows us to pay attention to the context surrounding the use of critical sections in a design.

We have decided to present the problem and its solution in a slightly unusual way.  Instead of presenting the problem as a specification to implement, we present a design that we need to refine and see that mutual exclusion arises as one of the components of a solution.

\todo{ present a slightly different development than in \cite{thesis/hudon2011} for the sake of modularity }

\subsection{Initial Specification}

The design from where we start features a shared variable, $x$, and a set of actions that each implement an atomic transformation of $x$.

assume $\st p.i$
\[  \quant{ [] }{ i }{}{ p.i  \1\rightarrow x, p.i := f.i.x, \cfalse } \]
\[ \tag{A0.i} p.i  \1\rightarrow x, p.i := f.i.x, \cfalse \]
\[  \quant{ [] }{ i }{}{ A.i } \]
\todo{format}

\subsection{Introducing Mutual Exclusion and Strong Fairness}

We would like to refine it to implement each of the $f.i$ in isolation and possibly in more than one step.  For this to be possible, it would be necessary to create a protection around $x$ so that only one process can assign to it at any given time.

New assumption 

provided
\[  \tag{HA0} x = k \1\un \neg w.i  \]
which can be translated in $  q.x \1\un \neg w.i  $
for any $q$
then
\[ \tag{A1.i}  p.i \land w.i  \1\rightarrow x, p.i := f.i.x, \cfalse \]

This is a schedule strengthening and it yields the properties below:
\begin{align}
	\tag{P0}
	p.i \1\mapsto p.i \land w.i \\
	\tag{S0}
	p.i \land w.i \1\un \neg p.i
\end{align}

We can take care of P0 with:\todo{P0 satisfied }

\[ \tag{B0.i} p.i \1\land \neg w.i \2\rightarrow w.i := \ctrue \]

%\[ \qforall{j, k}{i \neq j}{ x = k \1\un \neg w.j \1\uin A.i } \]

Let's cross-compare the effect of A.i with each other's hypothesis (HA0.j with $i \neq j$).  We need to make sure that $w.j$ is false before we execute A.i.

new invariant

\[ \tag{J0} \qforall{i,j}{i \neq j}{\neg w.i \1\lor \neg w.j }  \]

Which is why we need mutual exclusion.  Now to take care of it, we have to compare it to B0 since it's the only action assigning to $w$.

\todo{put short analysis}

And this is why we need strong fairness

\subsection{Eliminating Strong Fairness}

%%% Local Variables: 
%%% mode: latex
%%% TeX-master: "progress"
%%% End: 
