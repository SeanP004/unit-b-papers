\section{Conclusion}
\label{sec:conclusion}

We presented in this paper \unitb, a formal method inspired by \eventB
and \unity.  Our method allows systems to be developed gradually via
refinement and support reasoning about both safety and liveness
properties.  An important feature of \unitb is the notion of coarse-
and fine-schedules for events.  Standard weak- and strong-fairness
assumptions can be expressed using these event schedules.  We
proposed refinement rules to manipulate the coarse-
and fine-schedules such that liveness properties are
preserved.  We illustrated \unitb by developing a signal control
system.

A key observation in \unitb is the role of event scheduling regarding
liveness properties being similar to the role of guards regarding safety
properties.  Guards prevent events from occurring in some unsafe state so
that safety properties won't be violated; similarly, schedules ensure the
occurrence of events in order to satisfy liveness properties.  Another
key aspect of \unitb is the role of progress properties during
refinement.  Often, to ensure the validity of a refinement, one needs
to prove some progress properties which (eventually) can be
implemented (satisfied) by some scheduled events.

\paragraph{Related work}
% !TEX root = progress-llncs.tex
% \section{Related Work}
% \label{sec:related-work}

%%%%% Event-B
% Our \unitb method is inspired by \eventB and \unity.  We borrow the
% idea of refining specifications (transition systems) from \eventB and
% extending it to address liveness properties.  
\unitb and \eventB differ mainly in the scheduling
assumptions.  In \eventB, event executions are assumed to
satisfy the \emph{minimal progress} condition: as long
as there are events that are enabled, one will be chosen non-
deterministically. %
%\todo{Son: (to Simon) Check the definition of minimal progress}%
Given this assumption, certain liveness properties can be proved
for \eventB models such as \emph{progress} and
\emph{persistence}~\cite{hoang11:_reason_liven_proper_event_b}.
However, this work does not discuss how the refinement notion can be
adapted to preserve liveness properties.  Moreover, the
minimum progress assumption is often either too weak to prove
liveness properties or, when it's not, make the proofs needlessly 
complicated.

%   Often, one needs to have stronger assumptions
% such as weak- or strong-fairness.
%
%%%%% UNITY
% Our temporal logic reasoning is inspired by the \unity logic.
% Operators such as $\tr$ and $\un$ are defined within
% \unity~\cite{DBLP:books/daglib/0067338}.  What we have done is to give
% definition to these operators using computation calculus and to prove
% several properties related to these operators.  The main difference
% between \unitb and \unity is in the method.  In \unitb, we gradually
% refine the model (the transition system) in a property-preserving
% manner.  In \unity, a specification is essentially a property and a
% program is a transition system.  In \unitb, we unify
% the notions of specifications and programs, allowing them to be
% evolved together during refinement.
%
%%%%% The notion of fairness
% A key important feature of \unitb is the introduction of the notion of
% coarse-schedule and fine-schedule.  They are more general than the
% standard weak- and strong-fairness assumptions that has been used in
% many methods including \unity and
% TLA+~\cite{DBLP_books_aw_Lamport2002}.  As illustrated by our example
% in Section~\ref{sec:second-refinement} and
% Section~\ref{sec:third-refinement}, in some cases, event scheduling
% information can be captured quite naturally using coarse- and
% fair-schedules, while this would not be straightforward to be captured
% by weak- or strong-fairness assumptions.
%
%%%%% TLA+
TLA+\cite{DBLP_books_aw_Lamport2002} is another formal method based on
refinement and which supports liveness properties.  The
execution of a TLA+ model is also captured as a formula with safety
and liveness sub-formulae.  However, refinement of the liveness part
in TLA+ involves calculating explicitly the fairness assumptions of the
abstract and concrete models.  This is not practical in our
opinion for developing realistic systems in general.  The
lack of practical rules for refining the liveness part in TLA+ might be
rooted in the view of the author of TLA+ concerning the
\emph{unimportance of liveness}~\cite[Chapter
8]{DBLP_books_aw_Lamport2002}.  In our opinion, liveness
properties are as important as safety properties to design
correct systems.

%%% Local Variables: 
%%% mode: latex
%%% TeX-master: "progress"
%%% End: 


\paragraph{Future work}
%%%%% Data refinement
Currently, we only consider superposition refinement in \unitb where
variables are retained during refinement.  More generally, variables
can be removed and replaced by other variables during refinement (data
refinement). We are working on extending \unitb to provide rules for data
refinement.

%%%%% Decomposition/Composition 
Another important technique for coping with the difficulties in
developing complex systems is composition/decomposition and is already
a part of methods such as \eventB and \unity.  We intend to investigate
on how this technique can be added to \unitb, in particular, the role
of event scheduling during composition/decomposition.

%%%%% Tool support
Given the close relationship between \unitb and \eventB, we are
looking at extending the supporting Rodin
platform~\cite{abrial10:_rodin} of \eventB to accomodate \unitb.  We
expect to generate the corresponding proof obligations according to
different refinement rules such that it can be verified using the
existing provers of Rodin.


%%% Local Variables: 
%%% mode: latex
%%% TeX-master: "progress"
%%% End: 
