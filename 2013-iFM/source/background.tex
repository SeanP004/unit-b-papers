% !TEX root = progress-llncs.tex

\section{Background: Computation Calculus}
\label{sec:background}

This section gives a brief introduction to computation calculus, based
on \cite{Dijkstra:1998p1128}.  Let $\State$ be the state space: a
non-empty set of ``states''.  Let $\Computation$ be the computation
space: a set of non-empty (finite or infinite) sequences of states
(``computations'').  The set of computation predicates $\CPred$ 
is defined as follows.
%
\begin{Definition}[Computation Predicate]
  $\CPred = \Computation \rightarrow \BOOL$, i.e. functions from
  computations to Booleans.
\end{Definition}

The standard boolean operators of the predicate calculus are lifted,
i.e. extended to apply to $\CPred$. For example, assuming $s, t \in
\CPred$ and $\Comp \in \Computation$, we have,\footnote{
  In this paper, we use $f.x$ to denote the result of applying a
  function $f$ to argument $x$.  Function application is
  left-associative, so $f.x.y$ is the same as $(f.x).y$.
}%
%
%\begin{eqnarray} 
%(s \implies t).\Comp & \Wide{\eqv} & (s.\Comp \implies
%t.\Comp) \label{eq:comp-impl} \\ 
%\qforall{i}{}{s.i}.\Comp & \Wide{\eqv} &
%\qforall{i}{}{s.i.\Comp}~. \label{eq:comp-forall}
%\end{eqnarray} 

  \begin{minipage}{0.45\linewidth}
    \begin{eqnarray}
(s \implies t).\Comp  \Wide{\eqv}  (s.\Comp \implies
t.\Comp) \label{eq:comp-impl} 
    \end{eqnarray}
  \end{minipage}
  \hfill
  \begin{minipage}{0.45\linewidth}
    \begin{eqnarray}
\qforall{i}{}{s.i}.\Comp  \Wide{\eqv}
\qforall{i}{}{s.i.\Comp}~. \label{eq:comp-forall}
    \end{eqnarray}
  \end{minipage}
\\

The everywhere-operator quantifies universally over all
computations, i.e.
%
\begin{eqnarray} 
\ew{s} & ~\eqv ~& \qforall{\Comp}{}{s.\Comp} \label{eq:comp-ew}
\end{eqnarray}
%
%\todo{Son: (to Simon) Hmmm, I think you can do it better than me} %
Whenever there are no risks of ambiguity, we
shall use $s = t$ as a shorthand for $\ew{s \eqv t}$ for computation
predicates $s, t$.
%Equivalence between computation predicates $s$ and $t$, i.e., $s = t$
%is the same as $\ew{s \eqv t}$.

\begin{Postulate}
  \label{post:comp-pred-alg}
  $CPred$ is a predicate algebra.
\end{Postulate}
%
A consequence of \Post~\ref{post:comp-pred-alg} is that $\CPred$
satisfies all postulates for the predicate calculus as defined in
\cite{Dijkstra:1990:PCP:77545}.  In particular, $\ctrue$ (maps all
computations to $\True$) and $\cfalse$ (maps all computations to
$\False$) are the ``top'' and the ``bottom'' elements of the complete
boolean lattice with the order $\ew{ \_ \implies \_}$ specifying by
these postulates.  The lattice operations are denoted by various
boolean operators including $\land, \lor, \neg, \implies$, etc.


The predicate algebra is extended with sequential composition as follows.
%
\begin{Definition}[Sequential Composition]
  \begin{equation}
    \label{eq:comp-seq}
    (s;t).\Comp  \WIDE\eqv (\size \Comp = \infty \land s.\Comp) \wide\lor
    \qexists{n}{n < \size \Comp}{s.(\Comp \take n\! + \!1) \land t.(\Comp
      \drop n)}
  \end{equation}
%
where $\size$, $\take$ and $\drop$ denote sequence operations
`\emph{length}', `\emph{take}' and `\emph{drop}', respectively.
\end{Definition}
Intuitively, a computation $\tau$ satisfies $s\,;t$ if either it is an
infinite computation satisfying $s$, or there is a finite prefix of $\tau$
(i.e. $\tau \take n\! + \!1$) satisfying $s$ and the corresponding suffix $\tau
\drop n$ (which overlaps with the prefix on one state) satisfying $t$.  
% Sequential composition operator satisfies the following postulate.
% %
% \begin{Postulate}
%   \label{post:comp-seq-comp}
%   $;$ is universally disjunctive in its first and positively
%   disjunctive in its second argument.
% \end{Postulate}
% %

In the course of reasoning using computation calculus, we make use of
the distinction between infinite (``eternal'') and finite
computations.  Two constants $\E, \F \in \CPred$ have been defined for
this purpose. 
\begin{Definition}[Eternal and Finite Computations] For any predicate $s$,\\
  \begin{minipage}{0.4\linewidth}
    \begin{eqnarray}
      \E & = & \ctrue;\cfalse \label{eq:comp-eternal} \\
      \F & = & \neg \E \label{eq:comp-finite}
    \end{eqnarray}
  \end{minipage}
  \hfill
  \begin{minipage}{0.4\linewidth}
    \begin{eqnarray}
      \textrm{$s$ is eternal} & \, \eqv \, & \ew{s \implies \E} \label{eq:comp-s-eternal} \\
      \textrm{$s$ is finite} & \, \eqv\, & \ew{s \implies \F} \label{eq:comp-s-finite}
    \end{eqnarray}
  \end{minipage}
\end{Definition}
Given $\F$ the temporal ``eventually'' operator (i.e., $\diamondsuit$)
can be formulated as $\F;s$.  The ``always'' operator $\G$ is defined
as the dual of the ``eventually'' operator.
\begin{Definition}[Always Operator]
  $\G s = \neg(\F;\neg s)$, for any predicate $s$
\end{Definition}
Important properties of $\G$ are that it is strengthening and
monotonic.  For any predicates $s$ and $t$, we have:

  \begin{minipage}{0.3\linewidth}
	\begin{align}
	  \label{eq:g-strengthen}
	  \ew{\G s \limp s}~, \\
	  \nonumber
	\end{align}
  \end{minipage}
  \hfill
  \begin{minipage}{0.5\linewidth}
    \begin{align}
      \label{eq:g-monotonic}
      \ew{s \limp t} \Wide{&\limp} \ew{\G s \limp \G t}~, \\ 
	\ew{ \G (s \limp t) &\Wide{\limp} (\G s \limp \G t) }~.
    \end{align}
  \end{minipage}
%\begin{equation}
%	% \text{monotonicity with \textbf{G}}
%\end{equation}

A constant $\one$ is defined as (left- and right-) neutral element for
sequential composition.
\begin{Definition}[Constant $\one$] For any computation $\tau$, $\one.\tau  \Wide{\eqv}  \size \tau = 1$
\end{Definition}

\paragraph{State Predicates}
In fact, $\one$ is the characteristic predicate of the state space.
Moreover, we choose not to distinguish between a single state and the
singleton computation consisting of that state, which allows us to identify
predicates of one state with the predicates that hold only for singleton
computations.  Let us denote the set of state predicates by 
$\SPred$.
\begin{Definition}[State Predicate] For any predicate $p$,
    $p \in \SPred \Wide{\eqv} \ew {p \implies \one}$.
\end{Definition}

A consequence of this definition is that $\SPred$ is also a complete
boolean lattice with the order $\ew{ \_ \implies \_}$, with $\one$ and
$\cfalse$ being the ``top'' and ``bottom'' elements.  It inherits all
the lattice operators that it is closed under: conjunction,
disjunction, and existential quantification.  The other lattice
operations, i.e. negation and universal quantification, are defined by
restricting the corresponding operators on $\CPred$ to state
predicates.  We only use state predicate negation in this paper.
\begin{Definition}[State predicate negation $\spneg$] For
  any state predicate $p$,
  $\spneg p = \neg p \land \one$~.
\end{Definition}
% \begin{Theorem}[State Restriction] 
%   For all $s$ and all state
%   predicates $p$, we have
%   \begin{align}
%     p;s = p;\ctrue ~\land s \notag \\
%     s;p = s ~\land \ctrue;p \tag{State restriction}
%     \label{thm:state-restriction} 
%   \end{align}
% \end{Theorem}

For a state predicate $p$, the set of computations with the initial
state satisfying $p$ is captured by $p\,;\ctrue$: the weakest such 
predicate.  A special notation $\initially : \SPred \rightarrow \CPred$ 
is introduced to denote this predicate.
\begin{Definition}[Initially Operator] For any state predicate $p$,
  $\initially p = p\,;\ctrue$
\end{Definition}
% Later, for simplicity, we often make transformations by applying
% \Defn~\ref{def:comp-initially} implicitly.

This entails the validity of the following rule, which we will use
anonymously in the rest of the paper: for $p, q$ two \emph{state
  predicates}, $p \, ; q \wide= p \land q$.


An important operator in LTL is the ``next-time operator''.   This is
captured in computation calculus by the notion of atomic computations:
computations of length 2.  A constant $\X \in \CPred$ is defined for
this purpose.
\begin{Definition}[Atomic Actions] For any computation $\tau$ and
  predicate $a$,
  \begin{eqnarray}
    \X.\tau  & \WIDE\eqv & \size \tau = 2  \label{eq:comp-next} \\
    \textrm{$a$ is an atomic action} & \WIDE\eqv & \ew {a \implies \X} 
  \end{eqnarray}
\end{Definition}
Given the above definition, the ``next'' operator can be expressed
as $\X\,;s$ for arbitrary computation $s$.

% Important properties of $\X$
% related to non-singleton computations, finite and eternal computations
% are as follows.
% \begin{align}
%   \ew {\one \not\eqv \X;\ctrue}  \label{eq:comp-non-singleton} \\
%  \infiter{\X} = \E \label{eq:next-eternal} \\
% \finiter{\X} = \F \label{eq:next-finite}
% \end{align}
% We first define a constant $\s$ that will be used in our formulation
% later.
% \begin{Definition}[Constant $\s$]
%  \begin{equation}
% 	\ew{ \s ~~\equiv \one \lor \X }
% 		\label{sem:def:S}
% \end{equation} 
% \end{Definition}

%%% Local Variables: 
%%% mode: latex
%%% TeX-master: "progress"
%%% End: 
