% !TEX root = progress-llncs.tex
\section{The Unit-B Method}
\label{sec:contribution}
\newBmch[Mch]{M}
\newBevt[evt]{e}
%\newBcst[initpred]{i}

This section presents our contribution: the \unitb method which is
inspired by \eventB and \unity.
% methodology
Similar to \eventB, \unitb is aimed at the design of software systems
by stepwise refinement.  It differs from \eventB by the capability of
reasoning about progress properties and its refinement-order which
preserves liveness properties.  It also differs from \unity by
unifying the notions of programs and specifications, allowing
refinement of programs.  

\subsection{Syntax}
Similar to \eventB, in \unitb systems are modelled by a transition system,
where the state space is captured by variables $v$ and the transitions are
modelled by guarded events.  Furthermore, \unitb has additional 
assumptions on how the events should be scheduled.  Using
an \eventB-similar syntax, a \unitb event has the following form:
\begin{equation}
  \small
  \ubeventinline{\evt}{t}{\guard.t.v}{\csched.t.v}{\fsched.t.v}{\assignment.t.v.v'}~,\label{eq:ubevent}
\end{equation}
where $t$ are the parameters, $\guard$ is the \emph{guard}, $\csched$
is the \emph{coarse-schedule}, $\fsched$ is the \emph{fine-schedule},
and $\assignment$ is the \emph{action} changing state variables $v$.
The action is usually made up of several \emph{assignments}, either
deterministic ($\bcmeq$) or non-deterministic ($\bcmsuch$).
%  We use
%the short form
% \begin{equation}
%   \small
%   \ubeventinline{evt}{}{\guard.v}{\csched.v}{\fsched.v}{\assignment.v.v'}~,\label{eq:ubevent-no-par}
% \end{equation}
% for events without parameters, and
% \begin{equation}
%   \small
%   \ubeventinline{evt}{}{}{\csched.v}{\fsched.v}{\assignment.v.v'}~,\label{eq:ubevent-no-grd}
% \end{equation}
% for events without parameters and guard (the guard is $\one$).
An event $\evt$ with parameters $t$ stands for multiple events.
Each corresponds to several non-parameterised events $\evt.t$, one for
each possible value of the parameter $t$.  Here $\guard$, $\csched$,
$\fsched$ are state predicates.  An event is said to be
enabled when the guard $\guard$ holds.  The scheduling assumption of
the event is represented by $\csched$ and $\fsched$ as follows: if
$\csched$ holds for infinitely long and $\fsched$ holds infinitely
often then the event is carried out infinitely often.  An event
without any scheduling assumption will have its coarse-schedule
$\csched$ equal to $\cfalse$.  An event having only the
coarse-schedule $\csched$ will have the fine-schedule to be $\one$.
Vice versa, an event having only the fine-schedule $\fsched$ will
have the coarse-schedule to be $\one$.%
% \todo{Son: (to Simon) Please make sure this convention makes sense. It
% is different from your thesis.}%
% \todo{Simon: (to Son) On page 69 of my thesis, after equation (3.7), we
% see the exact mapping that you showed above}%

In addition to the variables and the events, a model has an
initialisation state predicate \init constraining the initial value of
the state variables.
%A special event containing only action is used as the initialisation
%of the model.  
All computations of a model start from a state satisfying the
initialisation and are such that, at every step, either one of its
enabled events occurs or the state is unchanged, and each computation
satisfies the scheduling assumptions of all events.
%\todo{
% Simon: (to Son) On terminology. Above, in the section ``then'', 
% you call the part assignment and here, you call it action. My suggestion
% would be to call each of the individual clauses like $x' = x + y$ ``assignments''
% and call their conjunction ``actions''. 
% Second point: the initialization should 
% simply be a satisfiable state predicate. If we make it into an event, we're 
% stripping it of everything but the state predicate which we encode as a
% peculiar binary relation $R$ over states: one where $\qforall{x,y}{}{x(R)z \eqv y(R)z}$  }%

Properties of \unitb models are captured by two types of properties:
\emph{safety} and \emph{progress} (liveness).

\subsection{Semantics} We are going to use computation calculus to
give the semantics of \unitb models.  Let $\Mch$ be a \unitb model
containing a set of events of the form~\eqref{eq:ubevent} and an
initialisation predicate $\init$.  
%The initialisation $\init$ characterizes the set
%of initial states, corresponds a state predicate $\initpred$.
Since the action of the event can be described by a before-after
predicate $\assignment.t.v.v'$, it corresponds to an atomic action
$\Action.t ~=~ \qforall{e}{}{\initially (e = v) ~\limp~ \X \, ;
  \assignment.t.e.v}$%
%\footnote{%
%  Son: We can define this as $\qforall{e}{}{\initially (e = v) \limp
%    \X;S(t,e,v)}$%
%}%
.  Given that an event $\evt.t$ can only be carried out when it is
enabled, the effect of each event execution can therefore be
formulated as follows: $\action.(\evt.t) = \guard.t\,;\,\Action.t$.
% \todo{Son: (to Simon) I changed $\guard$ to $\guard.t$}
% \todo{Simon: (to Son) Good, I adjusted the spacing to make clearer that $.$ binds tighter than $;$}
A special constant $\SKIP$ is used to denote the atomic action that
does not change the state.
\begin{Definition}[Constant $\SKIP$]
  $\SKIP.\tau  \Wide{\eqv}  \size \tau \!=\! 2 \wide\land \tau.0\! =\! \tau.1$,
  for all traces $\tau$ ($\tau.0$, $\tau.1$ denotes the first two
  elements of $\tau$).
\end{Definition}

% \todo{Son: (to Simon) I would like to use $=$ instead of $\ew{\eqv}$}
% \todo{Simon: (to Son) That can work. You have to be careful with $=$ 
% when you open and close $\ew{ } $ many times in a formula. My advice
% would be (0) to explicitly state that $=$ and $\ew{\eqv}$ are equivalent 
% notations and (1) to use $=$ only in the case where the $\ew{ \_ }$ would
% include all the formula at hand.}

The semantics of \Mch is given by a computation predicate $\execution.\Mch$
which is a conjunction of a ``safety part'' $\safety.\Mch$ and a
``liveness part'' $\liveness.\Mch$, i.e., 
\begin{equation}
  \ew{\execution.\Mch \Wide{\eqv} \safety.\Mch \land \liveness.\Mch}~.\label{eq:execution}
\end{equation}
A property represented by a formula $s$ is satisfied by \Mch, if
\begin{equation}
  \ew{\execution.\Mch \limp s}~.\label{eq:property}
\end{equation}

\subsubsection{Safety} Below, we define the general form of one step
of execution of model \Mch and the \emph{safety} constraints on its
complete computations.
%A step of model \Mch is captured by the following
%atomic action:
\begin{align}
  \ew{\step.\Mch  \Wide{&\eqv} \qexists{\evt, t}{\evt.t \in \Mch}{\action.(\evt.t)} \,\lor\, \SKIP} \label{eq:step} \\
%\end{equation}
%Together with the initialisation, $\safety.\Mch$ is defined as follows:
%\begin{equation}
  \ew{\safety.\Mch  \Wide{&\eqv}  \initially \init \land
    \G(\step.\Mch \, ; \,\ctrue)}
\end{align}
% \todo{Simon: (to Son) I would rather not differentiate between $\initpred$ and
% $\init$. I would also prefer to use the name $\init$ everywhere. $i$ seems more
% suitable as a bound variable.}
% 

Safety properties of the model are captured by \emph{invariance} properties
(also called \emph{invariants}) and by \emph{unless} properties. 

An invariant $I(v)$ is a state-properties that hold at every reachable state of the model.
In order to prove that $I(v)$ is an invariant of $\Mch$,
we prove that $\ew{\execution.\Mch \wide{\limp} \G \initially \! I}$.
In particular, we rely solely on the safety part of the model to prove
invariance properties, i.e., we prove $\ew{\safety.\Mch \wide{\limp} \G
  \initially \! I}$.  This leads to the well-known invariance principle.
\begin{equation}
  \begin{array}{ll}
    & \ew{\init \limp I} \wide{\land} 
    \ew{\qforall{\evt, t}{\evt.t \in \Mch}{I \, ;  \action.(\evt.t) ~\limp~ \X;I}} \\
    \implies & \\
    & \ew{\safety.\Mch \limp \G \initially \! I}
  \end{array}
  \tag{INV}
\end{equation}
Invariance properties are important for reasoning about the correctness
of the models since they limit the set of reachable states.  In
particular, invariance properties can be used as additional
assumptions in proofs for progress properties.

The other important class of safety properties is defined by the
\emph{unless} operator $\un$.
\begin{Definition}[$\un$ operator]  For all state predicates $p$ and
  $q$,
  \begin{equation}
    \label{eq:un-def}
    \ew{(p \un q) \WIDE{\eqv} \G (\initially p \wide\limp (\G \initially \! p)\,;(\one \lor \X)\,;\initially q)}
  \end{equation}
\end{Definition}
Informally, $p \un q$ is a safety property stating that if condition
$p$ holds then it will hold continuously until $q$ becomes true.
The formula $(\one \lor \X)$ is used in \eqref{eq:un-def} to allow
the last state where $p$ holds and the state where $q$ first holds to 
either be the same state or to immediately follow one another.
% \todo{Son: (to Simon) Is this explanation about $\un$ clear?}
% \todo{Simon: (to Son) Is this clearer?}
The following theorem is used for proving that a \unitb model
satisfies an unless property.
\begin{Theorem}[Proving an $\un$-property]
  \label{thm:unless}
  Consider a machine \Mch and property $p \un q$.  If
  \begin{equation}
    \label{eq:un-rule}
    \qforall{\evt,t}{\evt.t \in \Mch}{\, \G \!(\,(p \, \land \! \spneg q) ; \!
      \action.(\evt.t);\!\ctrue ~\,\limp\,~ \X ; \! (p \lor q);\! \ctrue \,)}
  \end{equation}
% \todo{Simon: (to Son) be careful, $;$ binds tighter than the 
% 	logical connectives so $s\,; (t \land u)$ and $s\, ; t \,\land\, u$
% 	are vastly different}
  then $\ew{\execution.\Mch \wide\limp p \un q}$
\end{Theorem}
\begin{proof}[Sketch]
  Condition \eqref{eq:un-rule} ensures that every event of \Mch either
  maintains $p$ or establishes $q$. By induction, we can see that the
  only way for $p$ to become false after a state where it was true is
  that either $q$ becomes true or that it was already true.
\end{proof}

\subsubsection{Liveness}
For each event of the form~\eqref{eq:ubevent}, its schedule
$\schedule.(\evt.t)$ is formulated as follows, where $\csched$ and
$\fsched$ are the event's coarse- and fine-schedule, respectively.
\begin{equation}
  \label{eq:schedule}
  \ew{\schedule.(\evt.t) \WIDE{\eqv} \G (\G \initially \!\csched
      \,\land\, \G \F\,;\initially \fsched  \wide{\limp} 
      \F\,;\fsched\,;\action.(\evt.t)\,;\ctrue)}~.
\end{equation}
To ensure that the event $\evt.t$ only occurs when it is enabled, we
require the following \emph{feasibility} condition:
\begin{equation}
  \label{eq:fis}
  \ew{\execution.\Mch \Wide{\limp} \G \initially \!(c \land f \, \limp\, g)}
  \tag{SCH-FIS}
\end{equation}

Our scheduling is a generalisation of the standard weak-fairness and
strong-fairness assumptions. The standard \emph{weak-fairness}
assumption for event $\evt$ (stating that if the event is enabled
infinitely long then eventually it will be taken) can be formulated
by using $\csched = \guard$ and $\fsched = \one$.
Similarly, the standard \emph{strong-fairness} assumption for $\evt$
(stating that if the event is enabled infinitely often then eventually
it will be taken) can be formulated by using $\csched = \one$ and
$\fsched = \guard$.
\begin{align}
  \ew{ \wf.(\evt.t)  &\equiv \G (\G \bullet \guard \1\implies
    \F;\action.(\evt.t);\ctrue) } \\
  \ew{ \strf.(\evt.t)  &\equiv \G (\G \F;\bullet \guard \1\implies \F;\action.(\evt.t);\ctrue) }
\end{align}



The liveness part of the model is the conjunction of the schedules for its
events, i.e.,
\begin{equation}
  \label{eq:liveness}
  \ew{\liveness.\Mch \Wide{\eqv} \qforall{\evt, t}{\evt.t \in \Mch}{\schedule.(\evt.t)}}~
\end{equation}

% semantics
% We will now discuss the semantics of the temporal logic we will be
% using, the semantics of our notion of program and the definition of
% refinement together with some rules.  All the semantics is expressed
% using computation calculus \cite{Dijkstra:1998p1128}.  For the sake of
% brevity, the proofs of the various properties of the semantics are
% omitted but they can be found in \cite{thesis/hudon2011}.

\subsection{Progress Properties}
\label{sec:progress-properties}
Progress properties are of the form $p \leadsto q$, where
$\leadsto$ is the leads-to operator.
\begin{Definition}[$\leadsto$ operator] For all state predicates $p$
  and $q$,
  \begin{equation}
    \label{eq:leadsto}
    \ew{(p \leadsto q) \WIDE{\eqv} \G (\initially p \wide{\limp} \F \initially \! q) }
  \end{equation}
\end{Definition}
In this paper, properties and theorems are often written without
explicit quantifications: these are universally quantified over all
values of the free variables occurring in them.

Important properties of $\leadsto$ are as follows. For state
predicates $p$, $q$, and $r$, we have:
%\begin{equation}
%  \label{eq:leadsto-one}
%  \ew{ p \leadsto \one}
%\end{equation}
% \todo{Simon: (to Son) Do you really need $p \leadsto \one$? We could
%   put disjunction instead}
\begin{align}
%  \label{eq:disjunction}
%  \ew{ \qexists{i}{}{p.i} \leadsto q \WIDE{&\eqv} \qforall{i}{}{p.i \leadsto q}
%  }\tag{Disjunction} \\
%  \label{eq:lhs-antimon}
%  \ew{ \G \initially\!(p \limp q) \WIDE{\limp} (\,(q \leadsto r) \wide{& \limp} (p \leadsto r)\,) }
%  \tag{LHS-Antimonotonicity}\\
  \ew{ (p \limp q) \WIDE{&\limp} (p \leadsto q)
  }\tag{Implication}\label{eq:implication} \\
  \ew{ (p \leadsto q) \land (q \leadsto r) \WIDE{&\limp} (p \leadsto r)
  }\tag{Transitivity}\label{eq:transitivity}
  \\
  \label{eq:37} 
  \ew{ (p \leadsto q)  \WIDE{&\eqv}  (p \,\land \! \spneg q \Wide{\leadsto} q)
  }\tag{Split-Off-Skip}
\end{align}
% \todo{Son: (to Simon) Think about the name for
%   \eqref{eq:37}}

The main tool for reasoning about progress properties in \unitb is the \emph{transient operator} $\tr$. 
\begin{Definition}[$\tr$ operator] For all state predicate $p$, $\ew{\tr p \Wide{\eqv} \G \F \, ;\initially\! \spneg p}$.
%  \begin{equation}
%    \ew{\tr p \Wide{\eqv} \G \F \, ;\initially\! \spneg p}~.\label{eq:transient}
%  \end{equation}
\end{Definition}
$\tr p$ states that state predicate $p$ is infinitely often false.
The relationship between $\tr$ and $\leadsto$
is as follows:
\begin{equation}
  p \wide{\leadsto} \spneg p  \WIDE{=} \one \wide{\leadsto} \spneg p \WIDE{=}
  \tr p~.
  \label{eq:trans-prop}
\end{equation}

The attractiveness of properties such as $\tr p$ is that we can
\emph{implement} these using a single event as follows.
\begin{Theorem}[Implementing $\tr$]
  \label{thm:transient} Consider a \unitb model \Mch and a transient
  property $\tr p$. We have
  $\ew{\execution.\Mch \limp \tr p}$, if there exists an event 
  \[\ubeventinline{\evt}{t}{\guard.t.v}{\csched.t.v}{\fsched.t.v}{\assignment.t.v.v'}~,\]
  that is to say $\execution.\Mch$ entails:
  \begin{equation}
  	\label{eq:LIVE}
  	\G (\G \initially \!c \land \G \F\,;\initially f \wide{\limp} \F\,; f\, ; \action.(\evt.t) ) ~ ,
  	\tag{LIVE}
  \end{equation}
  and parameter $t$ such that $\evt.t \in \Mch$ and $\execution.\Mch$ entails each of the conditions below:
  \begin{equation}
    \G \initially\!(p \limp \csched)~,\label{eq:SCH}
    \tag{SCH}
  \end{equation}
  \begin{equation}
    \csched \leadsto \fsched~,\label{eq:OP}
    \tag{PRG}
  \end{equation}
  \begin{equation}
    \G (~ (p \land \csched \land \fsched)\,; \action.(\evt.t) \, ; \ctrue \wide{\wide\limp} \X\,; \initially \! \spneg p~)~.
    \tag{NEG}
    \label{eq:NEG}
  \end{equation}
\end{Theorem}
\begin{proof}
  In this case, $\G$ acts as an everywhere operator which allows us to
  prove $\F;\initially \spneg p$ instead of $\G \F;\! \initially \!
  \spneg p$.  Additionally, since $\ew{ \neg s \limp s \wide{\eqv} s
  }$ for any computation predicate $s$, we discharge our proof
  obligation by strengthening $\F\; ; \, \initially \! \spneg p$ to
  its negation, $\G \initially \! p$.
    \begin{calculation}
    	\F \, ; \initially\! \spneg p
    \hint{\follows}{ $\ew{ \F\,;\X \limp \F}$, aiming for \eqref{eq:NEG} }
    	\F \, ; \X \, ; \initially \! \spneg p
    \hint{\follows}{ \eqref{eq:NEG} }
    	\F\, ; (p \land c \land f) \, ; \action \, ; \ctrue
    \hint{\follows}{ computation calculus }
%    	\F\, ; (~(c \land f) ; \! \action ; \! \ctrue \wide{\land} \initially \!\!\, p~)
%    \hint{\follows}{ $\G$ is strengthening then persistence rule }
	\F ;\! f ; \! \action ; \! \ctrue \wide{\land} \G \initially \! c \wide{\land} \G \initially \! p
    \hint{\follows}{ \eqref{eq:LIVE}; $\G$ is conjunctive }
    	\G  \F ; \! \initially f \wide{\land} \G \initially\! c \wide{\land} \G \initially \! p
    \hint{=}{ \eqref{eq:OP} }
    	\G \initially \! c \wide{\land} \G \initially \! p
    \hint{=}{ $\G$ is conjunctive; \eqref{eq:SCH} }
    	\G \initially \! p
    \end{calculation}
%  We prove $\tr p$ under the assumption $\execution.\Mch$.
%
%  First, we prove that $\G \initially p \limp \G \initially \csched$
%  as follows.
%  \begin{Reason}
%    \Step{}{\execution.\Mch}
%    \StepR{$\limp$}{~\eqref{eq:SCH}}{
%      \G \initially(p \limp \csched)
%    }
%    \StepR{$\limp$}{$\G$ is monotonic \eqref{eq:g-monotonic}}{
%      (\G \initially p \limp \G \initially \csched)
%    } 
%  \end{Reason}
%  
%  Second, we prove
%  that $ (\G \initially \csched \limp \G \F;\initially \fsched)$ as
%  follows.
%  \begin{Reason}
%    \Step{}{\execution.\Mch}
%    \StepR{$\limp$}{\eqref{eq:OP}}{
%      \csched \leadsto \fsched
%    }
%    \StepR{=}{Definition \eqref{eq:leadsto}}{
%      \G (\initially \csched \limp \F;\initially \fsched)
%    }
%    \StepR{$\limp$}{$\G$ is strengthening}{
%      (\initially \csched \limp \F;\initially \fsched)
%    }
%    \StepR{$\limp$}{$\G$ is monotonic}{
%      (\G \initially \csched \limp \G \F;\initially \fsched)
%    }
%  \end{Reason}
%  Furthermore, we have $\G (\G \initially p \wide{\limp} \G
%      \F\,;(p \land \csched \land \fsched)\,;\action.(\evt.t)\,;\ctrue)$ according to the following
%      reasoning.
%  \begin{Reason}
%    \Step{}{\execution.\Mch}
%    \StepR{$\limp$}{Definition \eqref{eq:execution}}{\liveness.\Mch}
%    \StepR{$\limp$}{Definition
%      \eqref{eq:liveness}}{\schedule.(\evt.t)}
%    \StepR{$=$}{Definition \eqref{eq:schedule}}{
%      \G (\G \initially \csched
%      \land \G \F;\initially \fsched  \wide{\limp} \G
%      \F;\fsched;\action.(\evt.t);\ctrue)      
%    }
%    \StepR{$=$}{$\G \initially \csched \limp \G \F;\initially \fsched$}{
%      \G (\G \initially \csched \wide{\limp} \G
%      \F;\fsched;\action.(\evt.t);\ctrue)      
%    }
%    \StepR{$=$}{$\G \initially \csched$}{
%      \G (\G \initially \csched \wide{\limp} \G
%      \F;\csched \land \fsched;\action.(\evt.t);\ctrue)      
%    }
%    \StepR{$\limp$}{$\G \initially p \limp \G \initially \csched$}{
%      \G (\G \initially p \wide{\limp} \G
%      \F;\fsched;\action.(\evt.t);\ctrue)      
%    }
%    \StepR{$\limp$}{$\G \initially p$}{
%      \G (\G \initially p \wide{\limp} \G
%      \F;p \land \csched \land \fsched;\action.(\evt.t);\ctrue)      
%    }
%    \StepR{$\limp$}{~\eqref{eq:NEG}}{
%      \G (\G \initially p \wide{\limp} \G
%      \F;\X;\spneg p;\ctrue)
%    }
%    \StepR{$\limp$}{$\F;\X \limp \F$}{
%      \G (\G \initially p \wide{\limp} \G
%      \F;\spneg p;\ctrue)
%    }
%  \end{Reason}
%
%  Finally, we have
%  \begin{Reason}
%    \Step{}{(\G (\G \initially p \wide{\limp}  \G
%      \F;\spneg p;\ctrue) \Wide{\limp} \tr p)}
%    \StepR{=}{Definition of $\tr$ \eqref{eq:transient}}{
%      (\G (\G \initially p \wide{\limp}  \G
%      \F;\spneg p;\ctrue) \Wide{\limp} \G \F;\spneg p;
%      \true)
%    }
%    \StepR{$\follows$}{$\G$ is monotonic}{
%      ((\G \initially p \wide{\limp}  \G
%      \F;\spneg p;\ctrue) \Wide{\limp} \F;\spneg p; \true)
%    }
%    \WideStepR{=}{$\G \initially p  = \neg (\F;\spneg p; \true)$}{
%      ((\G \initially p \wide{\land}  \G
%      \F;\spneg p;\ctrue) \land \G \initially p \Wide{\limp} \F;\spneg p; \true)
%    }
%    \StepR{$\follows$}{Predicate logic}{
%      (\G \F;\spneg p;\ctrue \Wide{\limp} \F;\spneg p; \ctrue)
%    }
%    \StepR{=}{$\G$ is strengthening \eqref{eq:g-strengthen}}{
%      \ctrue
%    }
%  \end{Reason}
\end{proof}
(Due to space restriction, for the rest of this paper, we only present
sketch proofs of theorems. Detailed proofs are available
in~\cite{thesis/hudon2011}).

Condition \eqref{eq:SCH} is an invariance property. Condition
\eqref{eq:OP} is a progress property.  Condition \eqref{eq:NEG} states
that event $\evt.t$ establish $\spneg p$ in one step.  In practice,
often we design $\csched$ such that it is the same as $p$ and
$\fsched$ is $\one$ (i.e., omitting $\fsched$); as a result,
conditions \eqref{eq:SCH} and \eqref{eq:OP} are trivial.  Condition
\eqref{eq:NEG} can take into account any invariance property $I$ and
can be stated as $\ew{(I \land p \land \csched \land \fsched)\; ;
  \action.(\evt.t) ~\limp~ \X; \spneg p}$.

In general, progress properties can be proved using the following
\emph{ensure-rule}.  The rule relies on proving an unless property and
a transient property.
\begin{Theorem}[The ensure-rule] For all state predicates $p$ and $q$,
  \label{thm:ensure-rule}
  \begin{equation}
    \label{eq:ensure-rule}
    \ew{(p \un q) \land (\tr p \land \spneg q)  \WIDE\limp (p \leadsto q)}
  \end{equation}
\end{Theorem}
\begin{proof}[Sketch]
  $p \un q$ ensures that if $p$ holds then it will hold for infinitely long or
  eventually $q$ holds.  If $q$ holds eventually then we have $p
  \leadsto q$.  Otherwise, if $p$ holds for infinitely long and $\spneg q$
  also hold for infinitely long, we have a contradiction, since $\tr p \land
  \spneg q$ ensures that eventually $p \land \spneg q$ will be
  falsified.  As a result, if $p$ holds for infinitely long then eventually
  $q$ has to hold.
\end{proof}

\subsection{Refinement}
\label{sec:refinement}
\newBmch[cncMch]{N}

In this section, we develop rules for refining \unitb models such that
safety and liveness properties are preserved.  Consider a machine \Mch
and a machine \cncMch, \cncMch refines \Mch if
\begin{equation}
  \label{eq:ref}
  \ew{\execution.\cncMch \limp \execution.\Mch}~.
  \tag{REF}
\end{equation}
As a result of this definition, any property of \Mch is also satisfied
by \cncMch.  Similarly to \eventB, refinement is considered in \unitb
on a per event basis.  Consider an abstract event $\evt.t$ belong to
\Mch and a concrete event $\cncevt.t$ belong to \cncMch as follows.
\begin{gather*}
  \ubeventinline{\evt}{t}{\guard.t.v}{\csched.t.v}{\fsched.t.v}{\assignment.t.v.v'}\label{eq:absevt} \\
  \ubeventinline{\cncevt}{t}{\cncguard.t.v}{\cnccsched.t.v}{\cncfsched.t.v}{\cncassignment.t.v.v'}\label{eq:cncevt}
\end{gather*}
We have $\cncevt.t$ is a refinement of $\evt.t$ if
\begin{align}
  \label{eq:evt-safety}
  \ew{\execution.\cncMch ~\limp~ (\action.(f.t) \limp \action.(e.t))}~, \textrm{and}
  \tag{EVT-SAF}\\
  \label{eq:evt-live}
  \ew{\execution.\cncMch ~\limp~ (\schedule.(f.t) \limp \schedule.(e.t))}
  \tag{EVT-LIVE}
\end{align}
A similar rule is presented for the initialisation.  The proof that
\cncMch refines \Mch (i.e., \eqref{eq:ref}) given conditions such as
\eqref{eq:evt-safety} and \eqref{eq:evt-live} is left out.  A special
case of event refinement is when the concrete event \cncevt is a new
event.  In this case, \cncevt is proved to be a refinement of a special
\SKIP{} event which is unscheduled and does not change any abstract
variables.

Condition \eqref{eq:evt-safety} leads to similar proof obligations in
\eventB such as \emph{guard strengthening} and \emph{simulation}.  We
focus here on expanding the condition \eqref{eq:evt-live}.  The
subsequent theorems are related to concrete event
\cncevt~\eqref{eq:cncevt} and abstract event \evt~\eqref{eq:absevt},
under the assumption that condition \eqref{eq:evt-safety} has been
proved.  They illustrate different ways of refining event scheduling
information: \emph{weakening the coarse-schedule}, \emph{replacing the
  coarse-schedule}, \emph{strengthening the fine-schedule}, and
\emph{removing the fine-schedule}.

\begin{Theorem}[Weakening the coarse-schedule]
  Given $\cncfsched = \fsched$. If
  \begin{center}
    $\ew{\execution.\cncMch \WIDE\limp \G \initially(\csched \limp
      \cnccsched)}$ \WIDE{then} $\ew{\execution.\cncMch \limp
      (\schedule.(\cncevt.t) \limp \schedule.(\evt.t))}$~.
  \end{center}

\end{Theorem}
\begin{proof}[Sketch]
  The coarse-schedule is at an anti-monotonic position within the definition of
  $\schedule$.
\end{proof}

\begin{Theorem}[Replacing the coarse-schedule]
  \label{thm:ref-rep-crs}
  Given $\cncfsched = \fsched$.  If
  \begin{align}
    \ew{\execution.\cncMch &\limp \csched 
         \wide\leadsto \cnccsched}\label{eq:crs} \\
    \label{eq:unless}
    \ew{\execution.\cncMch &\limp \cnccsched ~\un~ \spneg \csched}~,
  \end{align}
  then $\ew{\execution.\cncMch \limp (\schedule.(\cncevt.t) \limp \schedule.(\evt.t))}$
\end{Theorem}
\begin{proof}[Sketch]
  Conditions~\eqref{eq:crs} and~\eqref{eq:unless} ensures that if $\csched$ holds
  then eventually $\cnccsched$ holds and it will hold for at least as long as $\csched$
  As a result, if
  $\csched$ holds for infinitely long, $\cnccsched$ also holds for infinitely long.  Hence
  the new schedule ensures that \cncevt occurs at least on those cases where \evt has to 
  occur.
\end{proof}

\begin{Theorem}[Strengthening the fine-schedule]
  \label{thm:strengthen-fns}
  Given $\cnccsched = \csched$. If
  \begin{align}
    \label{eq:fns}
    \ew{\execution.\cncMch \WIDE{&\limp} \G \initially(\cncfsched \limp
      \fsched)}~, \textrm{and} \\
    \label{eq:lts}
    \ew{\execution.\cncMch \WIDE{&\limp} \fsched \wide\leadsto \cncfsched}
  \end{align}
  then $\ew{\execution.\cncMch \limp (\schedule.(\cncevt.t) \limp \schedule.(\evt.t))}$.
\end{Theorem}
\begin{proof}[Sketch] We can prove  
  $\schedule.(\evt.t)$ under the assumptions 
  $\schedule.(\cncevt.t)$ and $\execution.\cncMch$ 
  by calculating from $\F \, ; (\csched \land \fsched) \, ; \action.(\evt.t)\, ; \ctrue$ 
  (the right hand side of $\schedule.(\evt.t)$) and applying 
  one assumption after the other (in this order \eqref{eq:fns}, 
  \eqref{eq:evt-safety}, $\schedule.(\cncevt.t)$, \eqref{eq:lts}) 
  to strengthen it to $\G \initially\! c \land \G \F\,;\initially\!f$ 
  (the right hand side of $\schedule.(\evt.t)$).
%     \begin{equation}
%       \schedule.(\cncevt.t) \label{eq:sfs:cfs}
%     \end{equation}
%   and $\execution.\cncMch$
%     \begin{calculation}
%     	\F\,;(\csched\land \fsched)\,;\action.\evt \, ; \ctrue
%     \hint{\follows}{ \eqref{eq:fns} }
%     	\F\,;(\csched\land \cncfsched)\,;\action.\evt \, ; \ctrue
%     \hint{\follows}{ \eqref{eq:evt-safety} }
%     	\F\,;(\csched\land \cncfsched)\,;\action.\cncevt \, ; \ctrue
%     \hint{\follows}{ \eqref{eq:sfs:cfs} and $\ew{ \F \eqv \F\,\F}$ }
%     	\G \initially\! \csched \land \G \F ; \! \F  ; \! \initially \cncfsched
%     \hint{\follows}{ \eqref{eq:lts} }
%     	\G \initially\! \csched \land \G  \F  ; \! \initially \fsched
%     \end{calculation}
% 
%%%Other proof
%  \begin{Reason}
%    \Step{}{\schedule.(\cncevt.t)}
%    \StepR{$=$}{Definition (\ref{eq:schedule})}{
%      \G (\G \initially \cnccsched \land \G \F;\initially
%      \cncfsched \limp \G\F;\cncfsched;\action.(\cncevt.t);\ctrue)
%    }
%    \StepR{=}{\eqref{eq:lts}}{
%      \G (\G \initially \cnccsched \limp \G\F;\cncfsched;\action.(\cncevt.t);\ctrue)
%    }
%    \StepR{$\limp$}{\eqref{eq:fns}}{
%      \G (\G \initially \cnccsched \limp \G\F;\fsched;\action.(\cncevt.t);\ctrue)
%    }
%    \StepR{$\limp$}{assumption $\action.(f.t) \limp \action.(e.t)$}{
%      \G (\G \initially \cnccsched \limp \G\F;\fsched;\action.(\evt.t);\ctrue)
%    }
%    \StepR{$\limp$}{$\cnccsched = \csched$}{
%      \G (\G \initially \csched \limp \G\F;\fsched;\action.(\evt.t);\ctrue)
%    }
%    \StepR{$\limp$}{Predicate logic}{
%      \G (\G \initially \csched \land \G \F;\fsched \limp \G\F;\fsched;\action.(\evt.t);\ctrue)
%    }
%    \StepR{$=$}{Definition (\ref{eq:schedule})}{
%      \schedule.(\evt.t)
%    }
%  \end{Reason}
\end{proof}

\begin{Theorem}[Removing the fine-schedule]
\label{thm:remove-fns}
  Given $\cnccsched = \csched$ and $\cncfsched = \one$. If
  \begin{equation}
    \label{eq:remove-fns}
    \ew{\execution.\Mch \WIDE\limp \G \initially (\csched \limp \fsched)}
  \end{equation}
  then $\ew{\execution.\cncMch \limp (\schedule.(\cncevt.t) \limp \schedule.(\evt.t))}$.
\end{Theorem}
\begin{proof}[Sketch]
  Condition \eqref{eq:remove-fns} ensures that when $\csched$ holds
  for infinitely long, $\fsched$ holds for infinitely long, hence we can remove the
  fine-schedule $\fsched$, i.e., replaced it by $\one$.
\end{proof}
%\subsection{Temporal Properties}
\label{sec:temporal-properties}
The temporal logic of \unity, on which ours is based, partitions temporal properties in two groups: progress properties and safety properties.
\todo{0.5 pages}

\subsubsection{Progress Properties}
\label{sec:progress-properties}
The main tool for expressing progress properties is the
\emph{leads-to} operator ($\mapsto$). For $p$ and $q$ two state
predicates, $p \mapsto q$ holds exactly for the computations in which
every state satisfying $p$ is eventually followed by a state (possibly
the same), satisfying $q$.
%
\begin{equation}
	\ew{ (p \mapsto q) \2\equiv \G (\bullet p \1\implies \F ; \bullet q) }
\end{equation}
%
The basic properties of $\mapsto$ make it reminiscent of the Hoare triple with the basic case, based on $\textbf{ensures}$ being analogous to the assignment rule, the disjunction rule to the rule for non-deterministic conditional statement, the transitivity rule to the rule for sequential composition and the induction rule to the rule for iteration.
\begin{itemize}
\item Transitivity
	\begin{equation}
		\ew{ (p \mapsto q)  \1\land (q \mapsto r) \2\implies  (p \mapsto r) }
			\label{sem:prog:lt:transitivity}
	\end{equation}
\item Disjunction \footnote{
%
In \cite{DBLP:books/daglib/0067338}, \cite{DBLP:journals/csur/Misra96}, the disjunction rule is presented as an implication (or more exactly, an inference rule) with $\qexists{i}{}{p.i} \mapsto q$ as the consequent.  However, in \cite{thesis/hudon2011}, we proved using a general formulation in computation calculus that the two parts of the rule are actually equivalent.  We see no reasons to refrain from using the equivalence. }
 %
	\begin{equation}
			\label{sem:prog:lt:disjunction} 
		\ew{ \qforall{i}{}{p.i \1\mapsto q }  \3{  \equiv }  \qexists{i}{}{p.i } \1\mapsto q  }
	\end{equation}
\item Induction
	\begin{equation}
%	\begin{array}{llll}
	\ew{  (p \1\land M = m  \2\mapsto (p \1\land M<m) \1\lor q) \3\implies (p \mapsto q)  }
%	\end{array} 
	\end{equation}
\end{itemize}


\subsubsection{Safety Properties}
\label{sec:safety-properties}
\todo{talk about $\G \_$}
\todo{$\s$ is undefined}
The main tool for expressing safety properties is the \emph{unless} operator ($\un$).  For $p$ and $q$ two state predicates, $p \un q$ holds exactly for the computations in which, once $p$ holds, it remains true for at least as long as $q$ is false.
%
\begin{equation}
	\ew{ (p \un q) \3\equiv \G (\bullet p \1\implies (\G \bullet p) ; \s ; \bullet q) }
\end{equation}
%
In the UNITY logic \cite{DBLP:books/daglib/0067338}, the unless operator is defined in terms of program execution but we have decided to separate the notions of program execution and temporal properties. 

\todo{list unless theorems}

\todo{ maybe we remove the proof of PSP }
\subsubsection{A Theorem: PSP}
As an example of what can be done with our computation calculus formulations, we offer the proof of the PSP theorem from \cite{DBLP:books/daglib/0067338} It is meant to show how straightforward a computation calculus proof is and how it relieves us from resorting to (meta-)proofs by induction over the structure of the proofs of the premise.
%
\begin{equation}
	\ew{ (\1{ p \land r \1\mapsto (q \land r) \lor b })  \3\follows  (p \mapsto q) \1\land (r \un b)  }
		\label{sem:prog:thm:psp}
\end{equation}

%%% Local Variables: 
%%% mode: latex
%%% TeX-master: "progress"
%%% End: 


%
\subsection{Program Semantics}
\label{sec:semantics}

%A program is defined as a set of guarded actions.

%\[ \ew{ \execution.\Prog \3\equiv \safety.\Prog \land \qforall{a}{a \in \Prog}{sch.a} } \]

%\subsubsection{Safety}

\[ \ew{ \safety.\Prog  \3\equiv  \G (\J \lor \qexists{a}{a \in \Prog}{ a });\ctrue } \]

where $\J$ is a special atomic action which does not change the state,
i.e. satisfying
\begin{align}
  & \ew{\J \implies \X} \notag \\
  & p;\J = \J;p~, \textrm{for all state predicate $p$}~.  \label{eq:skip}
\end{align}
\eqref{eq:saf} specifies that every computation starts from a
state satisfying $\InitPred$, and every two consecutive states are
supported by an atomic action in $A$ or the special atomic
action $\J$.

%%% Local Variables: 
%%% mode: latex
%%% TeX-master: "progress"
%%% End: 

\todo{defend that fairness is a realistic assumption}
\todo{...maybe on the ground that it is useful and we can refine it away}
\subsubsection{Scheduling}
For event based specification methods, there are three popular execution strategy. Minimal progress is a strategy for which an action of the system is chosen non-deterministically as long as some action is enabled. In principle, this is the strategy used in \eventB except that .

\paragraph{Weak and Strong Fairness}
With weak fairness, actions which remain enabled for infinitely long are chosen for execution infinitely often.  This is the strategy that was chosen for \unity.

Finally, strong fairness 
actions are chosen infinitely often for execution if they are enabled infinitely often.  It may not be suitable for direct implementation of programs but it provides an abstraction for specifying how contention over resources are settled without describing a mechanism but while ensuring that some progress properties hold. A brief treatment of strong fairness is presented in \cite[sect.\ 6.5.7]{DBLP:journals/csur/Misra96} but it doesn't use strong fairness as an abstraction.  Instead the book present some techniques to simulate strong fairness.  In \cite{DBLP:journals/fac/JutlaR97}, a rule for using strong fairness in proofs of progress is provided but it does not behave well with program refinement.

In \unitb, both weak fairness and strong fairness are covered directly. Minimal progress could be simulated but the need for doing so would be justified more by a particular execution platform than by the need for a particular abstraction.
%
\todo{2 pages}
\todo{distinguish between guard and schedules}
\begin{itemize}
\item Weak fairness.
	\begin{equation}
		\ew{ \wf.(g, A)  \2\equiv  \G (\G \bullet g \1\implies \G \F;g;A;\ctrue) }
	\end{equation}
\item Strong fairness.
	\begin{equation}
		\ew{ \strf.(g, A)  \2\equiv  \G (\G \F;\bullet g \1\implies \G \F;g;A;\ctrue) }
	\end{equation}
\end{itemize}
%
The necessity for putting together the notions of strong and weak fairness into one formulation of fair scheduling comes from the difficulty of using strong fairness alone with refinement.  Neither the notion of strong fairness nor that of weak fairness turn out to be any less useful as a consequence of the combination.

If we look at the definition of transient from \cite{DBLP:journals/csur/Misra96} i.e., 
	\[ \tr p \uin \Prog  \2\equiv 
			\qexists{g,a}{(g,a) \in A}{ \inv ( p \implies g ) \uin \Prog 
				\1\land \hoare{p}{a}{\neg p} } \]
we might be tempted to generalize it for strong fairness similarly to what was done in \cite{DBLP:journals/fac/JutlaR97} 
	\[ \tr p \uin \Prog  \2\equiv 
			\qexists{g,a}{(g,a) \in A}{ p \mapsto g \uin \Prog
				\1\land \hoare{p\land g}{a}{\neg p} } \]

The problem with the above definition is that, if we want to replace the guard of an action by $h$, we have to prove $p \mapsto h$ for all transient predicate relying on the action.  It would be hard to do, however, since transient predicates are not encoded in a program once they have been proven to hold and they are not among the properties that we want refinement to preserve. Preserving transient predicates would have the consequence of fixing a number of steps implementing a certain property which is not a commitment we wish to make early on.
\todo{guard? schedule?}

\todo{be careful: use "action" always or "event" always}
\todo{maybe rename $c$ as $cs$ and $f$ as $fs$ otherwise $f$ looks like a function}
\paragraph{Generalized Scheduling}
Instead, we distinguish between three kinds of guards that an action can have \emph{simultaneously}:
\begin{itemize}
\item The coarse schedule ($c$) has the role a guard has with respect to weak fairness: if it holds forever, the action will be executed infinitely often.  

\item The fine schedule ($f$) relates to strong fairness: if it holds infinitely often, the action will be executed infinitely often. 

\item The guard ($g$) prevents an action from being executed when it would violate an invariant or another safety property.  
\end{itemize}
It is important to see that while the two schedules make sure that an action is executed under certain circumstances, they say nothing of when it should \emph{not} be executed. On the other hand, the guard tells us when it is safe for the action to be executed but never prescribes that the action be executed. It only prevents it from being executed if the conditions are unfavorable.

We now present the formulation of our generalized scheduling assumption
\begin{itemize}
\item Generalized Scheduling
	\begin{equation}
			\label{eq:sch}
		\ew{ sch.(c, f, A)  \3\equiv  
			\G (\G \bullet c \1\land \G \F;\bullet f \2\implies \G \F;(c\land f);A;\ctrue) }
	\end{equation}
\end{itemize}

\todo{ feasibility }
\todo{ make sure that it is clear where the guard went }

For the sake of cohesion between the schedules and the guard, we require that the conjunction of the schedule be stronger than the guard.

\begin{equation}
\G \bullet (c \land f \1\implies g) 
\end{equation}
\todo{name proof obligations}


\subsubsection{Program Properties}
Finally, to reap the full benefits of the generalization, it is generally convenient to maintain $c \mapsto f$ as a property of the specification.
\todo{ nuance: a property holds of a specification if is a property of the specification or if it could be proved using the properties and the program }

\sout{we introduce coarse ($c$) and fine ($f$) schedules together and deal with them in separation. To implement $\tr p$, we proceed by satisfying the antecedent of \eqref{eq:sch} by requiring that $\G (p \implies c)$ and $c \mapsto f$ be true of the specification. This way, manipulating the schedule and the guard of an action becomes much simpler.}
\todo{together and in separation?}

\todo{magic scheduling?}

Our definition of transient predicate becomes:
\[ \ew{ (p \1\mapsto \spneg p) \2\follows ex.\Prop} \follows \tr p \uin \Prop \]
\begin{align*}
 \tr p \uin (\Prog, \Prop)  \2\equiv \qexists{a}{ a \in \Prog }{ \text{tra}.p.a.\Prop }
\end{align*}
\begin{align*}
	\text{tra}.p.a.\Prop \2\equiv 
	\quad	& \hoare{ p \land c \land f \land g }{ a }{ \neg p } \tag{FALS} \\
	\land ~~	& \ew{ \G \bullet( p \implies c) \2\follows ex.\Prop} \tag{EN} \\
	\land ~~ 	& \ew{ (c \mapsto f) \2\follows ex.\Prop } \tag{PR}
\end{align*}
\todo{check names with respect to Spec = (Prop, Prog)}
\todo{equation numbers}
\todo{where do we place $c \mapsto f$? \sout{in the implementation of $\mapsto$} or \underline{in def of $\tr$}}
\todo{deal with accessors (for schedules, guards, statement) in such a way that we can only name the ones that we need in any formula}

\todo{weave the definition of transient together with the scheduling detail}
\todo{1-2 pages}
\begin{itemize}
\item $\tr$
\item $\co$
\item invariants
\end{itemize}

In order to prove that a given program satisfies an arbitrary temporal property, we use program properties.  The definition of program properties are tied closely to the semantics of a program and provide a relation with the more general temporal properties.  However, unlike temporal properties, program properties are not all preserved by refinement.  For instance, transient predicates, which state that a certain goal is reached in one helpful step, are useful to implement leads-to properties but it is unnecessary to make sure that the given goal is only reached in one step: it might be appropriate at one level of abstraction to take only one step to accomplish a goal but, as the refinement goes, it might become necessary to distinguish between various phases of that step.
\todo{ is there really enough room to make a distinction between co and unless? }

%%% Local Variables: 
%%% mode: latex
%%% TeX-master: "progress"
%%% End: 


%\subsection{Refinement}
\label{sec:refinement}
In sections \ref{sec:temporal-properties} and \ref{sec:semantics}, a semantics for temporal properties and for programs has been shown but nothing has been said yet about the notion of specification. Whereas \unity's specification corresponds to a set of temporal properties and that of \eventB correspond to something more restrictive than our notion of program, we choose to use both as our notion of specification: in \unitb, a specification is a pair made of program and a set of temporal properties.  This allows us to express a wide variety of behaviors and a large spectrum between the requirements and their implementation.

We define refinement between specifications $R$ and $S$ as:
%
\[  R \sqsubseteq S ~~\equiv  \ew{ ex.R ~\follows ex.S } \]
%
\todo{define execution function $ex$}

with the function $ex$ defined as follows:

\begin{align}
ex.(P, T) \3{&\equiv} saf.P \land prog.P \land \qforall{tp}{tp \in T}{tp} \\
prog.P \3{&\equiv} \qforall{a}{a \in P}{sch.a}
\end{align}

\todo{complete first sentence} that is to say that every trace satisfying $R$ also satisfies $S$.  It is a fairly obvious definition to use but, on its own, it is not very useful because it deals with complete traces.  We would prefer to deal with assertions and relations between pairs of states.  The definition however provides a good basis for designing refinement rules.

\todo{ talk about sets of actions and sets of properties }

\subsubsection{Schedule Replacement}

Using the following rule, we can refine a specification by substituting the coarse schedule of an action for another. 
The idea was taken from \cite{misra:notes:strength}.  However, since we decoupled the notion of guard and coarse and fine schedules, we don't require the schedule to be strengthened.
Formally, we can refine a specification containing the following action: 

\newcommand{\AAA}{\Bevt{A}}
	
\begin{Bcode}
  $
  \ubevent{\AAA0}{}{& g}{ & BA }{ & p}{}
  $
\end{Bcode}

by replacing the action by the following action and properties: \todo{ make sure it is clear that the properties are part of the spec rather proved about the spec }

\begin{minipage}{0.48\textwidth}
\begin{Bcode}
  $
  \ubevent{\AAA1}{}{& g}{& BA }{& q}{}
  $
\end{Bcode}
\end{minipage}
\begin{minipage}{0.48\textwidth}
\bf{properties}
\begin{align}
& p ~\mapsto q \tag{P1} \label{ref:rep:P1} \\
& q ~\un \neg p \tag{S1} \label{ref:rep:S1} \\
\inv{~ q \implies g } \tag{J1}  \label{ref:rep:J1}
\end{align}
\end{minipage}
\todo{ find a nice layout to show properties as part of specifications}

It is important to note that, although we can map the above properties  to programs, they are not side conditions of the refinement rule; they are a part of the refined specification and can be, in turn refined either to code or to other properties.

\todo{define the expression "to be scheduled forever"}
The role of the added properties (\eqref{ref:rep:P1} and \eqref{ref:rep:S1}) is to ensure that the concrete action $\AAA1$ stays scheduled forever at least in those cases where the abstract action $\AAA0$ stays scheduled forever.  
\eqref{ref:rep:P1} ensures that, if $p$ becomes true, possibly before remaining true forever, $q$ will some time later also become true.  
Then, thanks to \eqref{ref:rep:S1}, provided $p$ and $q$ hold together, $q$ will hold as long as $p$ holds.

\todo{talk about feasibility - for example, the schedule may impose the execution of action $\AAA$ whenever $p$ holds but it may turn out that $p$ is to weak to ensure safety.}
Finally, the purpose of \eqref{ref:rep:J1} is to ensure the feasibility of the schedule.

\subsubsection{Introduction of Strong Fairness}

As a result of the strengthening of a guard, it may become necessary to strengthen the coarse schedule of an action.  However, the related proof obligations may prove impossible to satisfy.  In that case, we can add a fine schedule instead which incurs an obligation easier to satisfy.

Given is the following action in a specification:

\begin{Bcode}
  $
  \ubevent{\AAA0}{}{& g}{ & BA }{ & p}{}
  $
\end{Bcode}

we can add a fine schedule by replacing the action by the following action and property: \todo{ make sure it is clear that the properties are part of the spec rather proved about the spec }

\begin{minipage}{0.48\textwidth}
\begin{Bcode}
  $
  \ubevent{\AAA1}{}{& g}{& BA }{& p}{& q}
  $
\end{Bcode}
\end{minipage}
\begin{minipage}{0.48\textwidth}
\begin{Bcode}
%\bf{properties}
%\begin{align}
%& p ~\mapsto q \tag{P1} \label{ref:isf:P1} \\
%& \inv{~ p \land q ~\implies g } \tag{J1}  \label{ref:isf:J1}
%\end{align}
\ubprops{
	\ubproperty{ & p ~\mapsto q}{P1}{ref:isf:P1} \\
	\ubproperty{& \inv{~ p \land q ~\implies g }}{J1}{ref:isf:J1}
}
\end{Bcode}
\end{minipage}
\todo{introduce the syntax}
\todo{introduce distinct labels for actions and properties?}

The change between $\AAA0$ and $\AAA1$ consists in strengthening the conditions under which $\AAA\_$ will be scheduled in.  In addition to its coarse schedule ($p$) being satisfied for infinitely long, its fine schedule ($q$) must be satisfied infinitely often.  \eqref{ref:isf:P1} ensures that one implies the other; if $p$ holds forever, \eqref{ref:isf:P1} will cause $q$ to hold infinitely often, hence making the scheduling assumptions equivalent between the two specifications.
\todo{ define "scheduling assumptions" }

\subsubsection{Elimination of Strong Fairness}

The elimination of strong fairness boils down to remove fine schedules. In an action with coarse schedule $p$ and fine schedule $q$, the fine schedule can be removed provided the following property holds of the refinement:


\begin{minipage}{0.48\textwidth}
\bf{properties}
\begin{align}
& \inv{~ p  ~\implies q } \tag{J1}  \label{ref:isf:J1}
\end{align}
\end{minipage}


This is not usually the case but it can be achieved by modifying either of the coarse schedule ---with the rule shown above--- or the fine schedule ---using rules from \cite{thesis/hudon2011}.  In the example, only the latter will be necessary.
\todo{ really? }

\subsubsection{Others}

As in Event-B, the non-determinism present in individual actions can be reduced and their guards can be strengthened.

It is also possible to add desired progress and safety properties and variables as one moves along a refinement strategy to introduce the various requirements 

and to refine individual temporal properties into actions using the rules shown in section \ref{sec:temporal-properties}.

\todo{what if we started with the simplest cases?}
\todo{what about the proof obligation due to safety on future events?}

%%% Local Variables: 
%%% mode: latex
%%% TeX-master: "progress"
%%% End: 


%\subsection{Method}
\label{sec:method}
Unlike in \eventB and \unity, where a development is a series of specifications where we prove the refinement relation between every consecutive pairs, in \unitb, a development is a series of application of refinement rules.  We start with an empty specification and apply refinement rules to introduce new elements or move towards a concrete design. Intermediate specifications are reached in the process but we don't need to formulate them completely as long as their components are clearly identified. 
\todo{$<$ 1 page}

%%% Local Variables: 
%%% mode: latex
%%% TeX-master: "progress"
%%% End: 


%%% Local Variables: 
%%% mode: latex
%%% TeX-master: "progress"
%%% End: 
