\begin{abstract}
  We present Unit-B, a formal method inspired by Event-B and UNITY,
  for designing systems via step-wise refinement preserving both
  safety and liveness properties.  In particular, we introduce the
  notion of coarse- and fine-schedules for events, a generalisation of
  weak- and strong-fairness assumptions. We propose proof rules for
  reasoning about progress properties related to the schedules.
  Furthermore, we develop techniques for refining systems by adapting
  event schedules such that liveness properties are
  preserved.  We illustrate our approach by an example to show that
  Unit-B developments can be guided by both safety and liveness
  requirements.

  \textbf{Keywords}: progress properties, refinement, fairness,
  scheduling, Unit-B.
% We introduce \unitb, a unification of \eventB and \unity

% \eventB \cite{DBLP:books/daglib/0024570} and \unity \cite{DBLP:books/daglib/0067338} are two methods that brought something new to the field of design of concurrent programs. \unity presents a temporal logic as a means of specifying programs together with simple rules for mapping temporal properties to programs and techniques for refining specification. On the other hand, \eventB provides a unification of the notion of program and specification and defines a refinement order that preserves safety properties but not liveness.  In both \eventB and \unity, the notion of program is taken to be a transition system with state variables.

% We introduce \unitb \cite{thesis/hudon2011} as a unification of \eventB and \unity and, in the process, introduce a treatment of strong fairness which is amenable to refinement. The result is a notion of specification which subsumes that of program and for which refinement preserves both liveness and safety. The properties of the semantics of \unitb are formalized and proved using R.M. Dijkstra's computation calculus.

% From a methodological point of view, in a development in \unitb, unlike developments in \eventB, it is not necessary to postpone the proof of liveness properties until the last refinement; they can be introduced when they make most sense, exactly as is the case for safety properties in \eventB.  Furthermore, we argue that the liveness aspect of a specification can dictate the direction that a design should take and that a liveness preserving refinement order now allows us to take advantage of that fact.

\end{abstract}

%%% Local Variables: 
%%% mode: latex
%%% TeX-master: "progress"
%%% End: 
