
\subsection{Program Semantics}
\label{sec:semantics}

%A program is defined as a set of guarded actions.

%\[ \ew{ \execution.\Prog \3\equiv \safety.\Prog \land \qforall{a}{a \in \Prog}{sch.a} } \]

%\subsubsection{Safety}

\[ \ew{ \safety.\Prog  \3\equiv  \G (\J \lor \qexists{a}{a \in \Prog}{ a });\ctrue } \]

where $\J$ is a special atomic action which does not change the state,
i.e. satisfying
\begin{align}
  & \ew{\J \implies \X} \notag \\
  & p;\J = \J;p~, \textrm{for all state predicate $p$}~.  \label{eq:skip}
\end{align}
\eqref{eq:saf} specifies that every computation starts from a
state satisfying $\InitPred$, and every two consecutive states are
supported by an atomic action in $A$ or the special atomic
action $\J$.

%%% Local Variables: 
%%% mode: latex
%%% TeX-master: "progress"
%%% End: 

\todo{defend that fairness is a realistic assumption}
\todo{...maybe on the ground that it is useful and we can refine it away}
\subsubsection{Scheduling}
For event based specification methods, there are three popular execution strategy. Minimal progress is a strategy for which an action of the system is chosen non-deterministically as long as some action is enabled. In principle, this is the strategy used in \eventB except that .

\paragraph{Weak and Strong Fairness}
With weak fairness, actions which remain enabled for infinitely long are chosen for execution infinitely often.  This is the strategy that was chosen for \unity.

Finally, strong fairness 
actions are chosen infinitely often for execution if they are enabled infinitely often.  It may not be suitable for direct implementation of programs but it provides an abstraction for specifying how contention over resources are settled without describing a mechanism but while ensuring that some progress properties hold. A brief treatment of strong fairness is presented in \cite[sect.\ 6.5.7]{DBLP:journals/csur/Misra96} but it doesn't use strong fairness as an abstraction.  Instead the book present some techniques to simulate strong fairness.  In \cite{DBLP:journals/fac/JutlaR97}, a rule for using strong fairness in proofs of progress is provided but it does not behave well with program refinement.

In \unitb, both weak fairness and strong fairness are covered directly. Minimal progress could be simulated but the need for doing so would be justified more by a particular execution platform than by the need for a particular abstraction.
%
\todo{2 pages}
\todo{distinguish between guard and schedules}
\begin{itemize}
\item Weak fairness.
	\begin{equation}
		\ew{ \wf.(g, A)  \2\equiv  \G (\G \bullet g \1\implies \G \F;g;A;\ctrue) }
	\end{equation}
\item Strong fairness.
	\begin{equation}
		\ew{ \strf.(g, A)  \2\equiv  \G (\G \F;\bullet g \1\implies \G \F;g;A;\ctrue) }
	\end{equation}
\end{itemize}
%
The necessity for putting together the notions of strong and weak fairness into one formulation of fair scheduling comes from the difficulty of using strong fairness alone with refinement.  Neither the notion of strong fairness nor that of weak fairness turn out to be any less useful as a consequence of the combination.

If we look at the definition of transient from \cite{DBLP:journals/csur/Misra96} i.e., 
	\[ \tr p \uin \Prog  \2\equiv 
			\qexists{g,a}{(g,a) \in A}{ \inv ( p \implies g ) \uin \Prog 
				\1\land \hoare{p}{a}{\neg p} } \]
we might be tempted to generalize it for strong fairness similarly to what was done in \cite{DBLP:journals/fac/JutlaR97} 
	\[ \tr p \uin \Prog  \2\equiv 
			\qexists{g,a}{(g,a) \in A}{ p \mapsto g \uin \Prog
				\1\land \hoare{p\land g}{a}{\neg p} } \]

The problem with the above definition is that, if we want to replace the guard of an action by $h$, we have to prove $p \mapsto h$ for all transient predicate relying on the action.  It would be hard to do, however, since transient predicates are not encoded in a program once they have been proven to hold and they are not among the properties that we want refinement to preserve. Preserving transient predicates would have the consequence of fixing a number of steps implementing a certain property which is not a commitment we wish to make early on.
\todo{guard? schedule?}

\todo{be careful: use "action" always or "event" always}
\todo{maybe rename $c$ as $cs$ and $f$ as $fs$ otherwise $f$ looks like a function}
\paragraph{Generalized Scheduling}
Instead, we distinguish between three kinds of guards that an action can have \emph{simultaneously}:
\begin{itemize}
\item The coarse schedule ($c$) has the role a guard has with respect to weak fairness: if it holds forever, the action will be executed infinitely often.  

\item The fine schedule ($f$) relates to strong fairness: if it holds infinitely often, the action will be executed infinitely often. 

\item The guard ($g$) prevents an action from being executed when it would violate an invariant or another safety property.  
\end{itemize}
It is important to see that while the two schedules make sure that an action is executed under certain circumstances, they say nothing of when it should \emph{not} be executed. On the other hand, the guard tells us when it is safe for the action to be executed but never prescribes that the action be executed. It only prevents it from being executed if the conditions are unfavorable.

We now present the formulation of our generalized scheduling assumption
\begin{itemize}
\item Generalized Scheduling
	\begin{equation}
			\label{eq:sch}
		\ew{ sch.(c, f, A)  \3\equiv  
			\G (\G \bullet c \1\land \G \F;\bullet f \2\implies \G \F;(c\land f);A;\ctrue) }
	\end{equation}
\end{itemize}

\todo{ feasibility }
\todo{ make sure that it is clear where the guard went }

For the sake of cohesion between the schedules and the guard, we require that the conjunction of the schedule be stronger than the guard.

\begin{equation}
\G \bullet (c \land f \1\implies g) 
\end{equation}
\todo{name proof obligations}


\subsubsection{Program Properties}
Finally, to reap the full benefits of the generalization, it is generally convenient to maintain $c \mapsto f$ as a property of the specification.
\todo{ nuance: a property holds of a specification if is a property of the specification or if it could be proved using the properties and the program }

\sout{we introduce coarse ($c$) and fine ($f$) schedules together and deal with them in separation. To implement $\tr p$, we proceed by satisfying the antecedent of \eqref{eq:sch} by requiring that $\G (p \implies c)$ and $c \mapsto f$ be true of the specification. This way, manipulating the schedule and the guard of an action becomes much simpler.}
\todo{together and in separation?}

\todo{magic scheduling?}

Our definition of transient predicate becomes:
\[ \ew{ (p \1\mapsto \spneg p) \2\follows ex.\Prop} \follows \tr p \uin \Prop \]
\begin{align*}
 \tr p \uin (\Prog, \Prop)  \2\equiv \qexists{a}{ a \in \Prog }{ \text{tra}.p.a.\Prop }
\end{align*}
\begin{align*}
	\text{tra}.p.a.\Prop \2\equiv 
	\quad	& \hoare{ p \land c \land f \land g }{ a }{ \neg p } \tag{FALS} \\
	\land ~~	& \ew{ \G \bullet( p \implies c) \2\follows ex.\Prop} \tag{EN} \\
	\land ~~ 	& \ew{ (c \mapsto f) \2\follows ex.\Prop } \tag{PR}
\end{align*}
\todo{check names with respect to Spec = (Prop, Prog)}
\todo{equation numbers}
\todo{where do we place $c \mapsto f$? \sout{in the implementation of $\mapsto$} or \underline{in def of $\tr$}}
\todo{deal with accessors (for schedules, guards, statement) in such a way that we can only name the ones that we need in any formula}

\todo{weave the definition of transient together with the scheduling detail}
\todo{1-2 pages}
\begin{itemize}
\item $\tr$
\item $\co$
\item invariants
\end{itemize}

In order to prove that a given program satisfies an arbitrary temporal property, we use program properties.  The definition of program properties are tied closely to the semantics of a program and provide a relation with the more general temporal properties.  However, unlike temporal properties, program properties are not all preserved by refinement.  For instance, transient predicates, which state that a certain goal is reached in one helpful step, are useful to implement leads-to properties but it is unnecessary to make sure that the given goal is only reached in one step: it might be appropriate at one level of abstraction to take only one step to accomplish a goal but, as the refinement goes, it might become necessary to distinguish between various phases of that step.
\todo{ is there really enough room to make a distinction between co and unless? }

%%% Local Variables: 
%%% mode: latex
%%% TeX-master: "progress"
%%% End: 
